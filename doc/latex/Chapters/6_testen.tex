\chapter{Testen}\label{ch:testen}

%% Machen Sie sich auf Basis Ihrer Überlegungen zur Qualitätssicherung Gedanken darüber, wie Sie die Erfüllung der Anforderungen möglichst automatisiert im Rahmen von Teststufen (Unit-Test, Komponententest, Integrationstest, Systemtest, Regressionstest und Abnahmetest) überprüfen werden.

\section{Testplan}\label{sec:testplan}

%% Definieren Sie Zeitpunkte für die jeweiligen Teststufen in Ihrer Projektplanung. Dazu können Sie die Meilensteine zu Hilfe nehmen. Überlegen Sie, wie die Test-Architektur der jeweiligen Teststufen aussehen. Verwenden Sie Testmethoden wie z.B. Grenzwertanalyse, 100% Zustandsabdeckung, 100% Transitionsüberdeckung, Pfadüberdeckung, Tiefensuche, Breitensuche, etc. Versuchen Sie, so gut wie möglich, Ihre Tests zu automatisieren.

\section{Testszenarien}\label{sec:testszenarien}

%% <konkrete Beschreibungen der Test Szenarien. Bei Bedarf die Tabelle vervielfältigen>

\section{Abnahmetest}\label{sec:abnahmetest}

%% 6.3	Abnahmetest
%% Leiten Sie die Abnahmebedingungen aus den Kunden-Anforderungen her. Dokumentieren Sie hier, welche Schritte für die Abnahme erforderlich sind und welches Ergebnis jeweils erwartet wird (Test Cases).

\section{Testprotokolle und Auswertungen}\label{sec:testprotokolle-und-auswertungen}

%% Hier fügen Sie die Test Protokolle bei, auch wenn Fehler bereits beseitigt worden sind, ist es schön zu wissen, welche Fehler einst aufgetaucht sind. Eventuelle Anmerkung zur Fehlerbehandlung kann für weitere Entwicklungen hilfreich sein.
%% Das letzte Testprotokoll ist das Abnahmeprotokoll, das bei der abschließenden Vorführung erstellt wird. Es enthält eine Auflistung der erfolgreich vorgeführten Funktionen des Systems sowie eine Mängelliste mit Erklärungen der Ursachen der Fehlfunktionen und  Vorschlägen zur Abhilfe
