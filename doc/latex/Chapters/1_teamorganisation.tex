\chapter{Teamorganisation}\label{ch:teamorganisation}

%%  Überlegen Sie, welche Regeln Sie für die Zusammenarbeit aufstellen wollen und welche Rollen
%% Sie im Team verteilen wollen.
%% Dokumentieren Sie diese hier zusammen mit weiteren Anmerkungen der Teamorganisation.


\section{Verantwortlichkeiten}\label{sec:verantwortlichkeiten}

%% Benennen Sie Verantwortliche innerhalb des Projekts (Projektleiter, Tester, Implementierer, etc.).
%% Auch hier ist eine Listen- oder Tabellendarstellung angebracht.
Die Verantwortlichkeiten gehen aus Tabelle~\ref{tab:verantwortlichkeiten} hervor.

\begin{table}[h]
    \begin{center}
        \begin{tabular}{ |c|c|c| }
            \hline
            Responsibility                     & Vorname & Name     \\
            \hline\hline
            Projektleitung                     & Justin  & Hoffmann \\
            \hline
            Requirements Manager               & Moritz  & Ohsten   \\
            \hline
            Testleitung, Configuration Manager & Hugo    & Protsch  \\
            \hline
            Programmierleitung                 & Jendrik & Stoltz   \\
            \hline
        \end{tabular}
    \end{center}
    \caption{Verantwortlichkeiten}
    \label{tab:verantwortlichkeiten}
\end{table}


\section{Absprachen}\label{sec:absprachen}

%% Listen Sie hier die Absprachen im Team auf, z. B. Jour Fixe, Kommunikation, Respond-Latenz, ....


\section{Repository-Konzept}\label{sec:repository-konzept}

%% Überlegen Sie sich, wie Sie das Repository und die Ordner organisieren wollen.
%% Welche Regeln wollen Sie beim Umgang mit Branches, Auslieferungen, Nachrichten an den Commits
%% usw. im Team einhalten?. Listen Sie diese Absprachen hier auf. Überlegen Sie auch, wie die
%% Arbeitsabläufe sein sollen bei der Umsetzung von Arbeitsaufträgen oder bei der Behebung von
%% jFehlern.
