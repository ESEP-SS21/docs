\chapter{Teamorganisation}\label{ch:teamorganisation}

%%  Überlegen Sie, welche Regeln Sie für die Zusammenarbeit aufstellen wollen und welche Rollen
%% Sie im Team verteilen wollen.
%% Dokumentieren Sie diese hier zusammen mit weiteren Anmerkungen der Teamorganisation.


\section{Verantwortlichkeiten}\label{sec:verantwortlichkeiten}

%% Benennen Sie Verantwortliche innerhalb des Projekts (Projektleiter, Tester, Implementierer, etc.).
%% Auch hier ist eine Listen- oder Tabellendarstellung angebracht.
Die Verantwortlichkeiten gehen aus Tabelle~\ref{tab:verantwortlichkeiten} hervor.

\begin{table}[h]
    \begin{center}
        \begin{tabular}{ |c|c|c| }
            \hline
            Responsibility                     & Vorname & Name     \\
            \hline\hline
            Projektleitung                     & Justin  & Hoffmann \\
            \hline
            Requirements Manager               & Moritz  & Ohsten   \\
            \hline
            Testleitung, Configuration Manager & Hugo    & Protsch  \\
            \hline
            Programmierleitung                 & Jendrik & Stoltz   \\
            \hline
        \end{tabular}
    \end{center}
    \caption{Verantwortlichkeiten}
    \label{tab:verantwortlichkeiten}
\end{table}


\section{Absprachen}\label{sec:absprachen}

%% Listen Sie hier die Absprachen im Team auf, z. B. Jour Fixe, Kommunikation, Respond-Latenz, ....


\section{Repository-Konzept}\label{sec:repository-konzept}

%% Überlegen Sie sich, wie Sie das Repository und die Ordner organisieren wollen.
%% Welche Regeln wollen Sie beim Umgang mit Branches, Auslieferungen, Nachrichten an den Commits
%% usw. im Team einhalten?. Listen Sie diese Absprachen hier auf. Überlegen Sie auch, wie die
%% Arbeitsabläufe sein sollen bei der Umsetzung von Arbeitsaufträgen oder bei der Behebung von
%% jFehlern.

\subsection{Organisation}\label{subsec:organisation}
%% Überlegen Sie sich, wie Sie das Repository und die Ordner organisieren wollen.

\subsection{Git Practices}\label{subsec:git}

\paragraph{Branching Strategie}
%% Welche Regeln wollen Sie beim Umgang mit Branches
Für unsere Branching-Strategie verwenden wir als Grundlage
\href{https://docs.gitlab.com/ee/topics/gitlab_flow.html}{GitLab Flow}.
Der default Branch ist der \textit{master}-Branch, es können keine Commits direkt auf diesem
gepusht werden.
Um Änderungen vorzunehmen wird ein Feature-Branch erstellt.
In diesem werden die nötigen Änderungen vorgenommen.
Sobald das Feature abgeschlossen ist, wird dieser wieder in den \textit{master}-Branch gemerged.
Die Feature-Branches sind vom Umfang jeweils möglichst klein zuhalten, dies ermöglicht einen
genaueren Review Prozess, der weiter in Abschnitt~\ref{sec:qualitaetssicherung} ausgeführt wird.

Für das Pflichtenheft besteht außerdem ein eigener Branch mit dem Namen \textit{latex}, um den
\textit{master}-Branch übersichtlicher und Code-basiert zu halten.
In diesem können direkt Änderungen vorgenommen werden, dies ist jedoch nur für kleine Fixes, wie
z.\ B.\ Rechtschreibung, Formatierung usw., vorgesehen.
Für neue Abschnitte, oder für das Ändern des Inhaltes bestehender Abschnitte ist ein neuer Branch
anzulegen, sodass die Änderung reviewt werden können.

\paragraph{Commits}
%% Welche Regeln wollen Sie mit Nachrichten an den Commits
Commits sind möglichst atomar zu halten:
So stellt jeder Commit genau eine vollständige Änderung, zum Beispiel das Beheben eines
Fehlers, das Hinzufügen einer Funktion o.\ Ä.\ dar.
Der Zustand des Repositories bzw.\ Codes soll optimalerweise nach jedem Commit funktional sein.
Dies hat den Vorteil, dass Änderung leicht rückgängig gemacht und mithilfe von
\textit{git cherry-pick} ausgewählt werden können.

Commit-Messages sind aussagekräftig zu wählen.
Es soll beschrieben werden was geändert wurde bzw.\ warum es geändert wurde.
Wie etwas geändert wurde, geht hingegen aus dem Inhalt des Commits selber hervor und soll somit
nicht erwähnt werden.
Wir habens uns darauf geeinigt die Commit-Messages in Englisch im Imperativ zu verfassen, sodass
diese mit Standard Commit-Messages von Git konsistent sind.

\paragraph{Auslieferungen}\label{subsec:auslieferungen}
%% Welche Regeln wollen Sie beim Umgang mit Auslieferungen
Für Auslieferungen benutzten wir Git Tags in Kombination mit GitLab Releases, das genaue Vorgehen
muss noch festgelegt werden.
Die zwei möglichen Varianten hierbei sind einmal wie in Gitflow beschrieben ein
Production-Branch, in den der \textit{master} gemerged wird und der den Stand der aktuellen Auslieferung
widerspiegelt.
Alternativ kann auch direkt im \textit{master}-Branch ein Commit getaggt werden.
