\chapter{Projektmanagement}\label{ch:projektmanagement}

%% In diesem Kapitel sollten organisatorische Punkte beschrieben und festgelegt werden.


%\section{Prozess}\label{sec:prozess}
%% Der Prozess wird durch die anderen sections ausreichend beschrieben.

%% Legen Sie den Prozess fest, nach dem Sie das Projekt umsetzen wollen.
%% Geben Sie ggf. grobe Schritte an, wie Planungsrunden, Sprints, oder ähnliches.


\section{Absprachen}\label{sec:absprachen}

\begin{itemize}
    \item Die Kommunikation läuft über unseren Discord Server, für dringende Angelegenheiten
    wird eine WhatsApp-Gruppe genutzt
    \item Zweimal pro Woche wird ein Meeting gehalten.
    Die Agenda wird in
    \href{https://git.haw-hamburg.de/ss21-esep-gruppe-2.3/esep/-/boards/2082?label_name[]=Protokoll}
    {GitLab in je einem Issue} geführt.
    Es ist ein tabellarisches Protokoll zu führen, welches die Ergebnisse festhält, dieses wird
    in das Issue gestellt.
    \item Wir verwenden den Google C++ Coding Style Guide
\end{itemize}

%% Listen Sie hier die Absprachen im Team auf, z. B. Jour Fixe, Kommunikation, Respond-Latenz, ....


\section{Projektplan}\label{sec:projektplan}

%% User Stories/Projektstrukturplan, Ressourcenplan, Zeitplan, Abhängigkeiten von Arbeitspaketen,
%% eventueller Zeitverzug, Visualisierung des Projektstandes, etc.
Wir verwenden einen agilen Ansatz für das Umsetzten des Projektes und setzen diesen auf GitLab um.
Dabei orientieren wir uns an dem
\href{https://about.gitlab.com/blog/2018/03/05/gitlab-for-agile-software-development/}{Guide
zur Verwendung von GitLab für agile Software Entwicklung}:
\begin{itemize}
    \item Die Tasks aus dem Projektplanung-Template wurden auf
    \href{https://git.haw-hamburg.de/groups/ss21-esep-gruppe-2.3/-/epics}{GitLab als Epics} angelegt.
    Es wird jeweils eine Verantwortlichkeit mithilfe eines Labels zugeordnet
    \item Falls Epics größere Unteraufgaben beinhalten, werden diese als Subepics formuliert
    \item Für die festen Phasen werden
    \href{https://git.haw-hamburg.de/groups/ss21-esep-gruppe-2.3/-/milestones}{Milestones} verwendet
    \item Für einzelne Aufgaben werden
    \href{https://git.haw-hamburg.de/groups/ss21-esep-gruppe-2.3/-/issues}{Issues}
    verwendet, die, soweit möglich, einem Epic und Milestone zugeordnet werden.
    In einem Issue kann mit Kommentaren über das Issue selber diskutiert werden.
    Über die Assignee Funktion können einem Issue beliebig viele Personen zugeteilt werden, die für
    für das Lösen dieses Issues verantwortlich sind.
\end{itemize}
Sobald an einem Issue gearbeitet wird, wird, falls ein eigener Branch nötig ist, ein Draft MR
erstellt.
In diesem sammeln sich alle Änderungen für dieses Issue.
Jegliche implementierungsspezifische Kommentare können direkt in Commits an den jeweiligen
Codezeilen hinterlassen werden und werden im MR angezeigt.
Bei Änderungsvorschlägen ist ein neuer Thread statt eines Kommentars zu eröffnen, sodass diese im MR
auf den ersten Blick über \glqq unresolved threads\grqq{} zu sehen sind.
Sobald der Branch gemerged werden soll, wird der MR als \textit{ready} markiert und das Label
\textit{workflow::pending review} zugewiesen, was den anderen
Teammitgliedern signalisiert, dass sie mit dem Reviewprozess, wie in
Abschnitt~\ref{sec:qualitaetssicherung} beschrieben, beginnen können.


\section{Qualitätssicherung}\label{sec:qualitaetssicherung}

Die Qualität sowohl der erstellten Dokumentation, als auch des Codes
wird mithilfe eines Review bzw.\ Approval Prozesses bei MR sichergestellt.
Für einen Merge in den Master-Branch werden drei Approvals benötigt.
Für einen Merge in den Latex-Branch werden zwei Approvals benötigt.
Die Anzahl an nötigen Approvals kann bei jedem MR bearbeitet werden.
Wir haben uns darauf geeinigt, die Anzahl nicht herunter zu setzen,
sondern im Einzelfall zu erhöhen, falls es nötig erscheint.
Jeder aus dem Team kann einen MR approven.

Wenn ein MR approved wird, übernimmt der Approver neben dem Implementierer die volle Verantwortung
über die Korrektheit und Fehlerfreiheit der Änderungen.
Somit ist der Code Zeile für Zeile durchzugehen und auf Fehler sowie Abweichung von
unseren Codingstyle-Guidelines zu überprüfen.


%% Überlegen Sie, wie Sie Qualität in Ihrem Projekt sicherstellen wollen.
%% Listen Sie die Maßnahmen hier auf. Beachten Sie, dass diese Maßnahmen für die unterschiedlichen
%% Artefakte und Ebenen entsprechend unterschiedlich sein können.
