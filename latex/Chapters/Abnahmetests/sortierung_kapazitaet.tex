\abntest{Sortierung, Kapazität}{
    \refreq{1},
    \refreq{2},
    \refreq{3},
    \refreq{5},
    \refreq{6},
    \refreq{13}, %<- TODO auslagern
    \refreq{18},
    \refreq{30},
    \refreq{38},
    \refreq{39},
    \refreq{47}
}{
    \item Das System befindet sich im Betriebszustand und hat noch keine \glspl{workpiece} sortiert
    \item Falls nicht anders spezifiziert soll sichergestellt werden, dass
    \begin{enumerate}
        \item die \glspl{rampe}kapazität während der Tests nicht ausgeschöpft wird
        \item sich die \glspl{workpiece} während der Tests nicht überschlagen
    \end{enumerate}
}
    \label{abntest-sortierung-kapazitaet}

    \begin{ablauf}{Korrekte Reihenfolge}
        \item \Glspl{workpiece} in korrekter Reihenfolge einlegen:
        \begin{enumerate}[noitemsep, nolistsep]
            \item \gls{workpiece_metall}
            \item \gls{workpiece_bohrung}
            \item \gls{workpiece_flach}
            \item \gls{workpiece_metall}
        \end{enumerate}
    \end{ablauf}

    \erwartungText
    Am Ende von \gls{anlage} 2 kommen die \Glspl{workpiece} in
    unveränderter Reihenfolge an (\refreq{2}, \refreq{3})

    \begin{ablauf}{Falsche Reihenfolge}
        \item Eine \gls{anlage} mit \gls{weiche}, die andere mit \gls{ejector} ausstatten
        (\refreq{30}, \refreq{38}, \refreq{39})
        \item \Glspl{workpiece} in falscher Reihenfolge einlegen:\label{enm:reih1}
        \begin{enumerate}[noitemsep, nolistsep]
            \item \gls{workpiece_hoch} %x
            \item \gls{workpiece_metall}
            \item \gls{workpiece_metall} %x
            \item \gls{workpiece_bohrung}
            \item \gls{workpiece_hoch} %x
            \item \gls{workpiece_flach}
            \item \gls{workpiece_bohrung} %x
            \item \gls{workpiece_metall}
        \end{enumerate}
        \item \Glspl{rampe}kapazität von \gls{anlage} 1 manuell ausschöpfen
        \item \Glspl{workpiece} in gleicher Reihenfolge wie in
        Schritt~\ref{enm:reih1} einlegen\label{enm:reih2}
    \end{ablauf}

    \begin{erwartung}
        \item Bei Schritt~\ref{enm:reih1} und~\ref{enm:reih2}: Am Ende von \gls{anlage} 2 kommen
        \Glspl{workpiece} in folgender Reihenfolge an,
        alle anderen wurden aussortiert (\refreq{2}, \refreq{3})
        \begin{enumerate}[noitemsep, nolistsep]
            \item \gls{workpiece_metall}
            \item \gls{workpiece_bohrung}
            \item \gls{workpiece_flach}
            \item \gls{workpiece_metall}
        \end{enumerate}
    \end{erwartung}

    \begin{ablauf}{Beide \glspl{rampe} voll}
        \item \Glspl{rampe}kapazität von \gls{anlage} 1 manuell ausschöpfen\label{enm:rampe1-voll}
        \item \Glspl{rampe}kapazität von \gls{anlage} 2 manuell ausschöpfen\label{enm:rampe2-voll}
        \item \gls{workpiece_hoch} einlegen
        \item Warten, bis das \gls{workpiece_hoch} \gls{lb_sw} erreicht\label{itm:workpiece-erreicht-switch}
    \end{ablauf}

    \begin{erwartung}
        \item Nach Schritt~\ref{enm:rampe1-voll} blinkt die gelbe \gls{ampelled}
        von \gls{anlage} 1 (\refreq{5}, \refreq{13}) % TODO auslagern
        \item Nach Schritt~\ref{enm:rampe2-voll} blinkt die gelbe \gls{ampelled}
        von \gls{anlage} 2 (\refreq{5}, \refreq{13}) % TODO auslagern
        \item Nach Schritt~\ref{itm:workpiece-erreicht-switch},
         bleiben die \glspl{belt} beider \glspl{anlage} stehen (\refreq{6})
    \end{erwartung}

    \begin{ablauf}{Überschlagen von \glspl{workpiece}n}
        \item \gls{workpiece_metall} einlegen
        \item Direkt dahinter \gls{workpiece_bohrung} einlegen
        \item Nach der Übergabe von \gls{workpiece_metall} an \gls{anlage} 2 dieses manuel überschlagen
        \item Warten bis das \gls{workpiece_hoch}, welches vorher ein \gls{workpiece_metall} war,
        aussortiert wurde
        \item \gls{workpiece_metall} einlegen
    \end{ablauf}

    \erwartungText
    Am Ende von \gls{anlage} 2 kommt genau ein \gls{workpiece} von Typ \gls{workpiece_metall} an (\refreq{47}, \refreq{18})
