\chapter{Implementierung}\label{ch:implementierung}

%% Anmerkung: Nur wichtige Implementierungsdetails sollen hier erklärt werden.
%% Code-Beispiele (snippets) können hier aufgelistet werden, um der Erklärung zu dienen.
%% Welche Patterns haben Sie für Ihre Implementierung benutzt.
%% Anmerkung: Bitte KEINE ganze Programme hierhin kopieren!

\section{Basis für STMs}\label{sec:basis-fuer-stms}

\begin{figure}[h]
    \centering
    \includegraphics[scale = 0.5]{../out/diagrams/stage3/cd_stm_base_classes}
    \caption{Basisklassen für die Implementierung der STMs}
    \label{fig:cd-stm-base}
\end{figure}

In Abbildung~\ref{fig:cd-stm-base} ist ein Klassendiagramm der Basisklassen für STMs dargestellt.
Die vorliegende Architektur hat unter anderem das Ziel, die Logik unabhängig von GNX und
dem Dispatcher – damit unabhängig von QNET – zu gestalten.
Hierdurch wird das Testen der Logik ohne GNS-Services ermöglicht.

Der ExampleStmClient und ExampleTestClient sind im Klassendiagramm jeweils Beispielklassen für
das Umsetzten der Clients und TestClients.
Diese erzeugen jeweils den entsprechenden Context und übergeben sich selbst als IEventSender an diesen,
der diesen wiederum an den Zustand übergibt.
Die Zustände haben somit die Möglichkeit Events zu verschicken.



%TODO ^^^ text hierfür
