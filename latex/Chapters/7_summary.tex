\chapter{Evaluation}\label{ch:evaluation}

%% Führen Sie ein Teammeeting durch, in dem gesammelt wird, was gut gelaufen war,
%% was schlecht gelaufen war und was man im nächsten Projekt (z.B. im PO) besser machen will.
%% Listen Sie für die Aspekte jeweils mindestens drei Punkte auf.
%% Weitere Erfahrungen und Erkenntnisse können hier ebenso kommentiert werden,
%% auch Anregungen für die Weiterentwicklung des Praktikums.


\section{Bekannte Abweichung der Dokumentation von der Implementierung}
\label{sec:bekannte-abweichung-der-dokumentation-von-der-implementierung}

Die folgenden Events wurden als Hotfix hinzugefügt, und sind somit nicht in der Eventliste
(Anhang~\ref{ch:signalliste})
aufgeführt:
\begin{itemize}
    \item \textit{EVNT\_ACT\_CTRL\_Q1\_LED\_ON}
    \item \textit{EVNT\_ACT\_CTRL\_Q1\_LED\_OFF}
    \item \textit{EVNT\_ACT\_CTRL\_Q2\_LED\_ON}
    \item \textit{EVNT\_ACT\_CTRL\_Q2\_LED\_OFF}
\end{itemize}
Diese werden in der STM sort\_wrpc beim Betreten bzw.\ Verlassen von \textit{RampFull} verschickt,
um zu Signalisieren, an welcher der beiden Anlagen die Kapazität der Rampe ausgelastet ist.
In Abbildung~\ref{fig:stm_sort_workpiece} ist das Senden dieser Events ebenfalls nicht
aufgeführt.


\section{Limitationen des Systems}\label{sec:limitations}

\begin{itemize}
    \item Im \refreq{28} wird beschrieben, dass der Betrieb, nachdem ein \gls{estop}-Schalter gedrückt wurde,
    durch das Herausziehen aller \gls{estop}-Schalter und das Drücken des \gls{t_reset}
    \textit{an der Anlage, an dem auch der \gls{estop} gedrückt wurde}, neu gestartet werden kann.
    Die Implementierung weicht leicht davon ab.
    Der Betrieb kann durch Drücken eines beliebigen \glspl{t_reset} erneut aufgenommen werden.
    Die Sicherheit wird durch diese Limitation nicht gefährdet, da für das Verlassen des
    \gls{estop}-Modus beide \gls{estop}-Schalter herausgezogen werden müssen und anschließend
    bestätigt werden muss.
    Die Änderung wurde aufgrund der Implementierung des Dispatchers vorgenommen:
    Beim Broadcasten eines Events (hier des Tasters) kann nicht darauf geschlossen werden, an
    welcher \gls{anlage} dieses Event verursacht wurde.
    \item Wenn sich der Anlagenverbund aufgrund ausgeschöpfter Kapazität im Fehlerzustand befindet,
    die Kapazität beider Rampen ausgelastet ist und anschließen die Rampe der zweiten Anlage
    geleert wird, werden alle vorliegenden Warnungen gelöscht.
    Um diesen Fehler zu verhindern, müssen bei Vorliegen eines Fehlers stets beide \glspl{rampe}
    geleert werden.
    \item Der \gls{durchsatz} des Systems ist auf das Einlegen von \glspl{workpiece}n im
    Abstand von 20 cm beschränkt.
    Werden \glspl{workpiece} im Abstand von weniger als 20 cm eingelegt, kann das korrekte Sortieren
    nicht garantiert werden.
\end{itemize}


\section{Lessons Learned}\label{sec:lessons-learned}

\subsubsection{Repository}
Ein eingänglich ausführliches Einrichten einer guten Repository-Infrastruktur ist für den Workflow
Gold wert.
In enger Zusammenarbeit mit Git nach GitFlow konnten wir von Anfang an unseren Workflow vereinheitlichen
und optimieren.
Auch die Aufteilung des Repositories in seperate in Code- und Dokumentations-Verzeichnisse hat sich als
eine wertvolle Entscheidung herausgestellt.
Beide Repositories konnte parallel und unabhängig voneinander genutzt werden.
Außerdem hat uns beispielsweise das Erstellen von Pipelines, die sowohl unser PDF-Dokument, als auch unsere
UML-Diagramme kompilieren mehrere Stunden Arbeit erspart.


\subsubsection{Dynamische Dokumentation}
Viele Abläufe wie Refactoring und Bugfixing erfordern eine nachträgliche Anpassung in ihrer entsprechenden
Dokumentation.
Ein Verschlafen der Anpassung in den Diagrammen und Modellierungen kann die Veränderung zu einem
späteren Zeitpunkt umso schwieriger gestalten.
Um zu verhindern, dass modellierte Änderungen im Code verloren gehen oder implementierte Änderungen
in der Dokumentation nicht nachvollziehbar sind, wäre ein eingängliches Briefing aller Teammitglieder von Vorteil
gewesen, welches die Wichtigkeit einer dynamischen Dokumentation klarstellt.

\subsubsection{Anforderungsanalyse}
Die Anforderungsanalyse setzte sich aus zwei wesentlichen Prozessen zusammen:
Die Ermittlung der Anforderungen aus dem Dokument des Kunden und die anschließende Use-Case-Modellierung.
Zweiteres erforderte viele Iterationen und hat mehrere Wochen des Projektes in Anspruch genommen.
In unserer Anwendung beschreiben die Use-Cases ausschließlich Kommunikation mit Peripherie-Geräten,
wodurch sie anfangs sehr technisch und zu nah an der späteren Logik-Modellierung erstellt wurden.
Auch ein Use-Case-Diagramm in einem eingebetteten Echtzeit-System, wie dem unseren bietet wenig Mehrwert,
da aufgrund der Menge an Beziehungen zwischen Akteuren und Use-Cases eine unübersichtliche Entschuldigung
für ein Diagramm entsteht, das eigentlich Übersicht schaffen soll.
Hier hätte man sich viel Frustration und Zeitaufwand sparen können.
Die gelernte Lektion ist es, über Darstellungsmöglichkeiten der Anforderungen hinreichender zu reflektieren
und sich auf Wesentliches zu konzentrieren.
Keep it simple.

\subsubsection{Due-Dates und Absprachen}
Fünf Studenten haben unterschiedliche Tagesabläufe, Verpflichtungen und Produktivitätszeiten.
Dies muss bei der Planung stets mit beachtet werden.
Ein genaueres Einhalten von Due-Dates und Deadlines schafft zeitliche Puffer zwischen den Abgaben
und macht die Projektentwicklung für alle Beteiligten angenehmer.

% Positiv: Discord
% Negativ: Wenn 1-2 Leute was ueber nacht machen muessen alle ausreichend
%   gebrieft werden, Stichwort: Datamodel

\subsubsection{}