\chapter{Evaluation}\label{ch:evaluation}

%% Führen Sie ein Teammeeting durch, in dem gesammelt wird, was gut gelaufen war,
%% was schlecht gelaufen war und was man im nächsten Projekt (z.B. im PO) besser machen will.
%% Listen Sie für die Aspekte jeweils mindestens drei Punkte auf.
%% Weitere Erfahrungen und Erkenntnisse können hier ebenso kommentiert werden,
%% auch Anregungen für die Weiterentwicklung des Praktikums.

\section{Limitationen des Systems}\label{sec:limitations}

Im \refreq{28} wird beschrieben, dass der Betrieb, nachdem ein \gls{estop}-Schalter gedrückt wurde, durch Herausziehen allesr
\gls{estop}-Schalter und Drücken des \gls{t_reset}
der Anlage, an dem auch der \gls{estop} gedrückt wurde, neu gestartet werden kann. Die Implementierung weicht leicht davon
ab. Welcher der \glspl{t_reset} zum erneuten Starten des Sytems benutzt wird, spielt keine Rolle mehr. Grund für diese Änderung ist, dass das
System keine Unterscheidung darin macht, von welcher Anlage ein Event geschickt wird. Die Sicherheit wird dadurch aber nicht gefährdet, da die letzte Aktion vor Wiederaufnahme des Betriebs
stets durch den Benutzer geschieht.
Ist die Kapazität der \glspl{rampe} beider Anlagen ausgeschöpft und ein aufgelegtes \gls{workpiece} passt nicht in die
Sortierreihenfolge, wird das System in Fehlerzustand versetzt. Um dort rauszukommen, müssen stets beide \glspl{rampe} geleert
werden! Geschieht das nämlich nur bei einer \gls{rampe}, wird danach keine Warnung mehr ausgegeben, obwohl noch eine der
beiden \gls{rampe}n voll ist. Dies ist so realisiert worden, da der Durchsatz, falls immer nur eine \gls{rampe} geleert wird,
verschlechtert wird, weil das System aufgrund von ausgeschöpften \glspl{rampe} schneller wieder in Fehlerzustand gerät und
stoppt.
Der \gls{durchsatz} des Systems ist insofern beschränkt, dass \glspl{workpiece} nur im Abstand von 20 cm sortiert werden können.
Liegen die Werkstücke zu nah beieinander auf dem \gls{belt}, kann es sein, dass es bei einem langsameren Aussortieren
der \gls{weiche} zu Stau kommt und die korrekte Funktion des Sysmtems nicht mehr gewährleistet werden kann.
Durch einige Tests hat sich der oben genannte Sicherheitsabstand als unkritisch erwiesen.

\section{Lessons Learned}\label{sec:lessons-learned}