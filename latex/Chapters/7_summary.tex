\chapter{Evaluation}\label{ch:evaluation}

%% Führen Sie ein Teammeeting durch, in dem gesammelt wird, was gut gelaufen war,
%% was schlecht gelaufen war und was man im nächsten Projekt (z.B. im PO) besser machen will.
%% Listen Sie für die Aspekte jeweils mindestens drei Punkte auf.
%% Weitere Erfahrungen und Erkenntnisse können hier ebenso kommentiert werden,
%% auch Anregungen für die Weiterentwicklung des Praktikums.


\section{Bekannte Abweichung der Dokumentation von der Implementierung}
\label{sec:bekannte-abweichung-der-dokumentation-von-der-implementierung}

Die folgenden Events wurden als Hotfix hinzugefügt, und sind somit nicht in der Eventliste
(Anhang~\ref{ch:signalliste})
aufgeführt:
\begin{itemize}
    \item \textit{EVNT\_ACT\_CTRL\_Q1\_LED\_ON}
    \item \textit{EVNT\_ACT\_CTRL\_Q1\_LED\_OFF}
    \item \textit{EVNT\_ACT\_CTRL\_Q2\_LED\_ON}
    \item \textit{EVNT\_ACT\_CTRL\_Q2\_LED\_OFF}
\end{itemize}
Diese werden in der STM sort\_wrpc beim Betreten bzw.\ Verlassen von \textit{RampFull} verschickt,
um zu Signalisieren, an welcher der beiden Anlagen die Kapazität der Rampe ausgelastet ist.
In Abbildung~\ref{fig:stm_sort_workpiece} ist das Senden dieser Events ebenfalls nicht
aufgeführt.


\section{Limitationen des Systems}\label{sec:limitations}

\begin{itemize}
    \item Im \refreq{28} wird beschrieben, dass der Betrieb, nachdem ein \gls{estop}-Schalter gedrückt wurde,
    durch das Herausziehen aller \gls{estop}-Schalter und das Drücken des \gls{t_reset}
    \textit{an der Anlage, an dem auch der \gls{estop} gedrückt wurde}, neu gestartet werden kann.
    Die Implementierung weicht leicht davon ab.
    Der Betrieb kann durch Drücken eines beliebigen \glspl{t_reset} erneut aufgenommen werden.
    Die Sicherheit wird durch diese Limitation nicht gefährdet, da für das Verlassen des
    \gls{estop}-Modus beide \gls{estop}-Schalter herausgezogen werden müssen und anschließend
    bestätigt werden muss.
    Die Änderung wurde aufgrund der Implementierung des Dispatchers vorgenommen:
    Beim Broadcasten eines Events (hier des Tasters) kann nicht darauf geschlossen werden, an
    welcher \gls{anlage} dieses Event verursacht wurde.
    \item Wenn sich der Anlagenverbund aufgrund ausgeschöpfter Kapazität im Fehlerzustand befindet,
    die Kapazität beider Rampen ausgelastet ist und anschließen die Rampe der zweiten Anlage
    geleert wird, werden alle vorliegenden Warnungen gelöscht.
    Um diesen Fehler zu verhindern, müssen bei Vorliegen eines Fehlers stets beide \glspl{rampe}
    geleert werden.
    \item Der \gls{durchsatz} des Systems ist auf das Einlegen von \glspl{workpiece}n im
    Abstand von 20 cm beschränkt.
    Werden \glspl{workpiece} im Abstand von weniger als 20 cm eingelegt, kann das korrekte Sortieren
    nicht garantiert werden.
\end{itemize}


\section{Lessons Learned}\label{sec:lessons-learned}