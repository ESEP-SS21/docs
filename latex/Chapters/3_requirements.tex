\chapter{Requirements und Use Cases}\label{ch:requirements-und-use-cases}

\section{Systemebene}\label{sec:systemebene}

%% Die Anforderungen aus der Aufgabenstellung sind nicht vollständig. Die Struktur der nachfolgenden Kapitel soll Sie bei der Strukturierung der Analyse unterstützen. Dokumentieren Sie die Ergebnisse der Analysen entsprechend.

\subsection{Stakeholder}\label{subsec:stakeholder}

%% Ermitteln Sie die Stakeholder für das Projekt und listen Sie diese hier auf.

\paragraph{Externe Stakeholder}
\begin{itemize}
    \item Auftraggeber
    \begin{itemize}
        \item Erfüllung aller spezifizierten Anforderungen
        \item Pünktliche Lieferung zum vorgegebenen Termin
        \item Verwendung der vorgegebenen Hard- und Software
    \end{itemize}
    \item Betreuer
    \begin{itemize}
        \item Erreichen der Ziele am Ende jeder Phase
        \item Möglichst vollständige Dokumentation als Rückmeldungsgrundlage
    \end{itemize}
    \item Benutzer
    \begin{itemize}
        \item System stellt keine Gefahr dar
        \item Information über Fehlerzustände
        \item Möglichst selten Eingreifen erforderlich
        \item Hoher Durchsatz
        \item Einfache Bedienung und Inbetriebnahme (Dokumentation)
    \end{itemize}
    \item Verwaltung TI-Labor
    \begin{itemize}
        \item Keine Beschädigung der Anlagen
    \end{itemize}
\end{itemize}

\paragraph{Interne Stakeholder}
\begin{itemize}
    \item Entwickler
    \begin{itemize}
        \item Gute Testbarkeit
        \item Einfache Erweiterbarkeit und Modularität
        \item Einheitliche Schnittstellen und Benennungen
        \item Dokumentation (im Code) für Fehlersuche und Teamarbeit
    \end{itemize}
\end{itemize}

\subsection{Anforderungen}\label{subsec:anforderungen2}

%% In der Aufgabenstellung sind Anforderungen an das System gestellt.
%% Arbeiten Sie diese hier auf und ergänzen Sie diese entsprechend der Absprachen mit dem Betreuer.
%% Achten Sie auf die entsprechende Atribuierung.
%% Berücksichtigen Sie auch mögliche Fehlbedienungen und Fehlverhalten des Systems.

\subsection{Systemkontext}\label{subsec:systemkontext2}

%% Use Cases werden aus einer bestimmten Sicht erstellt.
%% Dokumentieren Sie diese mittels Kontextdiagramm oder Use Case Diagramm.
%% Die Use Cases und Test Cases müssen zu der hier verwendeten Nomenklatur konsistent sein.

\subsection{Use Cases / User Stories}\label{subsec:use-cases-user-stories}

%% Dokumentieren Sie hier, welche Use Cases/ User Stories Sie auf der Systemebene implementieren müssen.
%% Die Test Cases sollen später zu den Use Cases/ User Stories konsistent sein.

%% < Hier kommt die genaue Beschreibung der Use.
%% Pro Anforderung eine Tabelle benutzen. Die Tabelle nach Belieben vervielfältigen. >

\begin{usecase}{1}{UseCase Name}{hoch}
\addCollum{Mainflow}{
\item[1)] do something
\item[2)] do this
\item[2a)] or tihs
}\addCollum{Alternate flow}{
\item[1)] do something
\item[2)] do this
\begin{itemize}
    \item[1)] Very long text: Lorem ipsum dolor sit amet, consectetur adipiscing elit,
    \item[2)] do this
\end{itemize}
\item[2a)] or tihs
}
\end{usecase}
\addTextCollum{Description}{
    Lorem ipsum dolor sit amet, consectetur adipiscing elit,
    sed do eiusmod tempor incididunt ut labore et dolore magna aliqua.
    Ut enim ad minim veniam, quis nostrud exercitation ullamco laboris
    nisi ut aliquip ex ea commodo consequat. Duis aute irure dolor in
    reprehenderit in voluptate velit esse cillum dolore eu fugiat nulla pariatur.
    Excepteur sint occaecat cupidatat non proident, sunt in culpa qui officia
    deserunt mollit anim id est laborum
}

% eine referenz zu einem UC sieht so aus:
siehe  \nameref{uc:1}.


\section{Systemanalyse}\label{sec:systemanalyse}

%% Ihr technisches System hat aus Sicht der Software bestimmte Eigenschaften.
%% Was muss man für die Entwicklung der Software in Struktur, Schnittstellen,
%% Verhalten und an Besonderheiten wissen?
%% Wählen Sie eine Kapitelstruktur, die am besten zur Dokumentation Ihrer Ergebnisse geeignet ist.

\subsection{Art des Systems}

Bei dem zu entwerfenden System handelt es sich um die Steuerung eines Festo-Transfersystems, welche auf einem in die Anlage integrierten BeagleBoneBlack Einplatinencomputer zu realisieren ist.
Daher handelt es sich um ein Embedded System. 
Für das Zu entwickelnde Softwaresystem bedeutet dies, dass es zur Steuerung der Prozesse sehr nah an der in der Anlage verbauten Hardware arbeiten muss. 
Ein weiterer wichtiger Punkt ist, dass das System mit einer weiteren gleichartigen Anlage über das Netzwerk kommunizieren muss. Dies ist ebenfalls im Entwurfsprozess zu berücksichtigen.

\subsubsection{Eigenschaften des BeagleBoneBlack und Grundlagen QNX}

Auf dem BeagleBoneBlack Einplatinencomputer läuft das Echtzeitbetriebssystem QNX, das Kommunikation zwischen Prozessen und Threads mittels Message Passing bevorzugt.
Dieses Message Passing kann auch über das Netzwerk erfolgen, sodass Messages auch unkompliziert an einen weiteren BeagleBoneBlack geschickt werden können. 
Die Architektur sollte sich daher auf Message-Passing stützen, gerade um die Kommunikation mit der anderen gleichartigen Anlage zu erleichtern.
Da der BeagleBoneBlack ein eigenes Stück Hardware mit eigenem Betriebssystem ist, muss für die Entwicklung die Entwicklungsumgebung QNX Momentics benutzt werden, die Debugging und Testing ermöglicht.
Die Kommunikation mit dem BeagleBoneBlack und dem Entwicklungsrechner erfolgt über die Netzwerkverbindung.

\subsection{Kommunikation mit der Hardware}

Die Ansteuerung der Aktorik und Sensorik der Anlage erfolgt direkt über die GPIOs des BeagleBoneBlack. 
Die Sensorik ist hierbei an GPIO0, die Aktorik des Transfersystems an GPIO1 und die LEDs des Bedienpanels an GPIO2.
Bei den GPIOs können Bits einzeln gesetzt und gelöscht werden und es können die aktuellen Werte ausgelesen werden.
Auf Veränderung an den GPIOs kann mittels pro Pin konfigurierbarer Interrupts (steigende Flanke, fallende Flanke, Level=0, Level=1) reagiert werden.
Interrupts lassen sich nach dem Auslösen für einen einstellbare Zeit ignorieren (Debouncing).
Um später nicht direkt mit den GPIOs interagieren zu müssen, sollte ein Hardware Abstraction Layer einfachere Schnittstellen zum Ansteuern der Aktorik und für das Reagieren auf die Sensorik bereitstellen.
Der Höhensensor ist am ADC des BeagleBoneBlack angeschlossen. Dieser kann über das Schreiben in ein bestimmtes GPIO-Register aktiviert werden und kann dann nach erfolgreicher Messung einen Interrupt auslösen.

\subsection{Besonderheiten beim Systemaufbau}

Bei genauerer Betrachtung des Systemaufbaus fallen folgende Eigenschaften und Zusammenhänge besonders auf. 
Sie sind bei der Verhaltensmodellierung zu berücksichtigen und könnten das Erkennen von Szenarien vereinfachen.

\subsubsection{Platzierung der Lichtschranke bei der Höhenmessung}

Die Lichtschranke an der Höhenmessung ist so platziert, dass sich die Mitte eines Werkstücks bei ihrer Unterbrechung genau unter dem Höhenmesser befindet. 
Auf diese Art lassen sich die unterschiedlichen Formen von Werkstücken mittels einer einzelnen Messung Unterscheiden. 
Die gemessenen Höhen bei unterschiedlichen Werkstückarten gehen aus Tabelle~\ref{tab:werkstuecke} hervor.

\begin{table}[h]
    \begin{center}
        \begin{tabular}{ |c|c| }
            \hline
            Form                     & Höhe in mm \\
            \hline\hline
            HOCH                 &  25,0-25,4\\
            \hline
            FLACH                     & 21 \\
            \hline
            LOCH               & 15,8-16,4 \\
            \hline
        \end{tabular}
    \end{center}
    \caption{Höhen der unterschiedlichen Werkstückarten}
    \label{tab:werkstuecke}
\end{table}

Ebenso fällt auf, dass der Abstand der Lichtschranke für die Höhenmessung vom Anfang des Förderbands ca. 25 cm beträgt, was für die Bestimmung des Abstands bei der Werkstückübergabe zwischen zwei Anlagen wichtig sein könnte.

\subsubsection{Der Aufbau im Bereich des Aussortiermechnaismus}

Im Transportweg der Werkstücke befindet sich an der Stelle an der im Falle einer Weiche diese spätestens für Durchlass öffnen muss und im Falle eines Auswerfers dieser das Werkstück auswerfen muss eine Lichtschranke.
Das Unterbrechen dieser Lichtschranke kann daher als Signal zum Starten des Aussortiervorgangs verstanden werden. 
Oberhalb des Werkstücks befindet sich in dieser Position auch der Metallsensor.
Bei einer Anlage mit Weiche ist wichtig, dass diese bei längerem Verharren im geöffneten Zustand beschädigt werden kann.

\section{Softwareebene}\label{sec:softwareebene}

%% Sie sollen Software für die Steuerung des technischen Systems erstellen.
%% Aus den Anforderungen auf der Systemebene und der Systemanalyse ergeben sich
%% Anforderungen für Ihre Software.
%% Insbesondere wird sich die Software der beiden Anlagenteile in einigen Punkten unterscheiden.
%% Dokumentieren Sie hier die Anforderungen, die sich speziell für die Software ergeben haben.

\subsection{Systemkontext}\label{subsec:systemkontext}

%% Wie sieht der Kontext Ihrer Software aus? Wie erfolgt die Kommunikation mit Nachbarsystemen?
%% Liste der ein- und ausgehenden Signale/Nachrichten.

\subsection{Anforderungen}\label{subsec:anforderungen}

%% Welche wesentlichen Anforderungen ergeben sich aus den Systemanforderungen für Ihre Software?
%% Berücksichtigen Sie auch mögliche Fehlbedienungen und Fehlverhalten des Systems.
