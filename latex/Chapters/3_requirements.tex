\chapter{Requirements und Use Cases}\label{ch:requirements-und-use-cases}

\section{Systemebene}\label{sec:systemebene}

%% Die Anforderungen aus der Aufgabenstellung sind nicht vollständig. Die Struktur der nachfolgenden Kapitel soll Sie bei der Strukturierung der Analyse unterstützen. Dokumentieren Sie die Ergebnisse der Analysen entsprechend.

\subsection{Stakeholder}\label{subsec:stakeholder}

%% Ermitteln Sie die Stakeholder für das Projekt und listen Sie diese hier auf.

\paragraph{Externe Stakeholder}
\begin{itemize}
    \item Auftraggeber
    \begin{itemize}
        \item Erfüllung aller spezifizierten Anforderungen
        \item Pünktliche Lieferung zum vorgegebenen Termin
        \item Verwendung der vorgegebenen Hard- und Software
    \end{itemize}
    \item Betreuer
    \begin{itemize}
        \item Erreichen der Ziele am Ende jeder Phase
        \item Möglichst vollständige Dokumentation als Rückmeldungsgrundlage
    \end{itemize}
    \item Benutzer
    \begin{itemize}
        \item System stellt keine Gefahr dar
        \item Information über Fehlerzustände
        \item Möglichst selten Eingreifen erforderlich
        \item Hoher Durchsatz
        \item Einfache Bedienung und Inbetriebnahme (Dokumentation)
    \end{itemize}
    \item Verwaltung TI-Labor
    \begin{itemize}
        \item Keine Beschädigung der Anlagen
    \end{itemize}
\end{itemize}

\paragraph{Interne Stakeholder}
\begin{itemize}
    \item Entwickler
    \begin{itemize}
        \item Gute Testbarkeit
        \item Einfache Erweiterbarkeit und Modularität
        \item Einheitliche Schnittstellen und Benennungen
        \item Dokumentation (im Code) für Fehlersuche und Teamarbeit
    \end{itemize}
\end{itemize}

\subsection{Anforderungen}\label{subsec:anforderungen2}

%% In der Aufgabenstellung sind Anforderungen an das System gestellt.
%% Arbeiten Sie diese hier auf und ergänzen Sie diese entsprechend der Absprachen mit dem Betreuer.
%% Achten Sie auf die entsprechende Atribuierung.
%% Berücksichtigen Sie auch mögliche Fehlbedienungen und Fehlverhalten des Systems.

\subsection{Systemkontext}\label{subsec:systemkontext2}

\begin{figure}
    \centering
    \includegraphics[width=\textwidth]{../out/diagrams/stage1/systemkontext.pdf}
    \caption{Systemkontext}
    \label{fig:systemkontext}
\end{figure}


Betrachte Abbildung \ref{fig:systemkontext}.
Die Systemsicht bringt umliegende Aktorik des Systemkontext und System selber miteinander in Beziehung.

\paragraph{Akteure}

\begin{itemize}
    \item Control\_Panel\\
    Steuerungspanel der Anlage.
    Sie stellt einen START-, STOP- und RESET-Button zur Verfügung.

    \item Conveyor\_Belt\\
    Auf Förderband werden Werkstücke durch das System bewegt.
    \item FTS\_2\\
    Die jeweils andere Förderanlage, mit der das System kommuniziert.
    \item E\_Stop\\
    Ein Button zum Auslösen des Emergency-Stops.
    \item Metal\_Detector\\
    Metall-Sensor zum Erfassen des Materials eines Werkstückes.
    \item Height\_Measurement\\
    Höhen-Sensor zum Erfassen des Materials eines Werkstückes.
    \item Stoplight\\
    Ampel, die durch verschiedene LED-Einstellungen Zustandsinformationen der Anlage anzeigt.
    \item Sorting\_Mechanism\\
    Das System besitzt allgemein einen Aussortier-Mechanismus.
    Dieser ist in einer von zwei Varianten in die Anlage eingebaut:
    \begin{itemize}
        \item Ejector\\
        Ein Auswerfer, der sich bei Stromzufluss ausfährt und so Werkstücke in die Rampe stoßen soll.
        Ohne Strom ist der Aussortiermechanismus offen.
        \item Switch\\
        Eine Weiche, die sich bei Stromzufluss öffnet und Werkstücke durchlässt.
        Ohne Strom ist der Aussortiermechanismus geschlossen und befördert Werkstücke langsam in die Rampe.
    \end{itemize}
    \item LightBarrier\\
    Das System besitzt diverse Lichtschranken als Trigger und zur Steuerung des Kontrollflusses
    \begin{itemize}
        \item LightBarrier\_Start\\
        Lichtschranke am Anfang eines Förderbandes.
        Die Unterbrechung dieser Lichtschranke setzt das Förderband in Bewegung.
        \item LightBarrier\_Ramp\\
        Lichtschranke an der Rampe, in der die Werkstücke aussortiert werden.
        Die kurze Unterbrechung dieser Lichtschranke kann als erfolgreiche Aussortierung interpretiert werden.
        Eine stetige Unterbrechung signalisiert eine volle Rutsche.
        \item LightBarrier\_Height\\
        Lichtschranke an der Position des Höhensensors.
        Die Unterbrechung dieser Lichtschranke startet eine Höhenmessung.
        \item LightBarrier\_Switch\\
        Lichtschranke vor dem Aussortiermechanismus der Anlage.
        Die Unterbrechung startet den Aussortieralgorithmus.
        \item LightBarrier\_End\\
        Lichtschranke am Ende eines Förderbandes.
        Die Unterbrechung dieser Lichtschranke kann das transferieren des Werkstückes an die nächste Anlage einleiten.
        Am Ende der zweiten Anlage werden bei Unterbrechung dieser Lichtschranke Werkstück-Informationen an der Konsole ausgegeben.
    \end{itemize}
\end{itemize}

%% Use Cases werden aus einer bestimmten Sicht erstellt.
%% Dokumentieren Sie diese mittels Kontextdiagramm oder Use Case Diagramm.
%% Die Use Cases und Test Cases müssen zu der hier verwendeten Nomenklatur konsistent sein.

\subsection{Use Cases / User Stories}\label{subsec:use-cases-user-stories}

%% Dokumentieren Sie hier, welche Use Cases/ User Stories Sie auf der Systemebene implementieren müssen.
%% Die Test Cases sollen später zu den Use Cases/ User Stories konsistent sein.

%% < Hier kommt die genaue Beschreibung der Use.
%% Pro Anforderung eine Tabelle benutzen. Die Tabelle nach Belieben vervielfältigen. >

\input{Chapters/UseCases/uc1}
\begin{usecase}{03}{Handle Lightbarrier\_Switch}{hoch}
    \addTextCollum{Description}{
        Bindeglied zwischen \nameref{uc:02}, \nameref{uc:06} und \nameref{uc:15},
    }
    \addTextCollum{Trigger}{
        Light\_Barrier\_Switch unterbrochen
    }
    \addCollum{Mainflow}{
        \item[1)] UC2 determine sorting action
        \item[2)] Sorting action ist discard\_workpiece \textbullet
        \item[3)] UC6 discard\_workpiece
    }
    \addCollum{Alternate flow 2}{
        \item at step 2) of Mainflow
        \item[2a)] Sorting action ist do\_not\_discard\_workpiece
        \begin{itemize}
            \item[2a1)] UC15 do\_not\_discard\_workpiece
            \item[2a1)] Ende des use cases
        \end{itemize}
    }
    % postconditions sind in den jeweiligen unter UCs gehandelt.
\end{usecase}
% eine referenz zu einem UC sieht so aus:
siehe  \nameref{uc:1}.


\begin{usecase}{12}{Handle Error}{hoch}
    \addTextCollum{Verwandte Requirements}{
    REQ-36, REQ-37
    }
    \addTextCollum{Description}{
    Dieser Use Case beschreibt die Fehlerzustände und Anzeige der roten LED beim Auftritt eines Fehlers
    Fehler müssen quittiert und behoben werden.
    }
    \addTextCollum{Actors}{
    Stoplight, Control\_Panel
    }\addTextCollum{Precondition}{
    Ein Fehler im System ist neu aufgetreten
    }
    \addTextCollum{Trigger}{
    Ein Fehler ist aufgetreten
    }
    \addCollum{Mainflow}{
    \item[1)] Das System geht in den Fehlerzustand astehend\_unqittiert
    \item[2)] Rote LED blinkt schnell(1Hz)
    \item[3)] Warte bis Reset-Taster gedrückt wird  \textbullet
    \item[4)] Fehler wurde acknowledged
    \item[5)] Fehlerzustand wechselt zu astehend\_qittiert
    \item[6)] Rote LED leuchtet dauerhaft
    \item[7)] Warte bis alle Fehler behoben sind
    \item[8)] System verlässt Fehlerzustand
    \item[9)] Rote LED wird ausgeschaltet
    }
    \addCollum{Alternate flow 1}{
    \item at step 3 of Mainflow
    \item[3a)] Fehler wurde behoben
    \begin{itemize}
        \item[3a1)] Fehlerzustand wechselt gegangen\_unqittiert
        \item[3a2)] Rote LED blinkt langsam(0,5Hz)
        \item[3a3)] Warte bis Reset-Taster gedrückt wird
        \item[3a4)] Fehler wurde acknowledged
        \item[3a5)] Zum Schritt 7 des main flow zurückkehren
    \end{itemize}
    }
    \addTextCollum{Postcondition}{
    Aufgetretener Fehler wurden behoben und System ist wieder im Normalbetrieb
    }
\end{usecase}

\begin{usecase}{07}{E-Stop}{hoch}
    \addCollum{Actors}{
    \item E\_Stop, FTS\_2, Stoplight, Control\_Panel, Sorting\_Mechanism, Conveyor\_Belt
    }
    \addCollum{Precondition}{}
    \addCollum{Triggers}{
        \item[1)] E\_Stop-Taster von FTS wird betätigt
        \item[1a)] E\_Stop-Schalter von FTS\_2 wird betätigt
    }
    \addCollum{Mainflow}{
        \item[1)] Der aktuelle Zustand sämtlicher Sensoren und Aktoren von FTS und FTS\_2 wird gespeichert
        \item[2)] Conveyor\_Belt, Sorting\_Mechanism werden von FTS und FTS\_2 abgeschaltet.
        \item[3)] Die rote LED am Stoplight an FTS und an FTS\_2 wird eingeschaltet.
        \item[4)] E\_Stop-Schalter von FTS herausziehen
        \item[5)]
        \item[2a)] or tihs
    }
    \addCollum{Alternate flow}{
        \item st step 4) of Mainflow
        \item[2a)] E\_Stop-Schalter von FTS\_2 wurde betätigt
        \begin{itemize}
            \item[2a)] E\_Stop-Schalter von FTS\_2 herausziehen
            \item[2b)] go to 5) of Mainflow
        \end{itemize}
    }
    \addTextCollum{Description}{
        Lorem ipsum dolor sit amet, consectetur adipiscing elit,
        sed do eiusmod tempor incididunt ut labore et dolore magna aliqua.
        Ut enim ad minim veniam, quis nostrud exercitation ullamco laboris
        nisi ut aliquip ex ea commodo consequat. Duis aute irure dolor in
        reprehenderit in voluptate velit esse cillum dolore eu fugiat nulla pariatur.
        Excepteur sint occaecat cupidatat non proident, sunt in culpa qui officia
        deserunt mollit anim id est laborum
    }
\end{usecase}

\begin{usecase}{05}{Receive Workpiece}{hoch}
    \addTextCollum{Verwandte Requirements}{
        REQ-7, REQ-24, REQ-26
    }
    \addTextCollum{Description}{
        In diesem Use-Case wird das Förderband einer Anlage gestartet, sobald sie ein Werkstück erreicht
    }
    \addTextCollum{Actors}{
        Conveyor\_Belt, LightBarrier\_Start
    }
    \addCollum{Precondition}{
    \item Die Anlage befindet sich im Betriebszustand
    }
    \addCollum{Trigger}{
    \item LightBarrier\_Start
    }
    \addCollum{Mainflow}{
    \item[1)] Zuweisung einer eindeutigen ID an das betroffene Werkstück
    \item[2)] Der Conveyor\_Belt ist in Bewegung
    }

    \addCollum{Alternative Flow 1}{
    \item at step 2) of Mainflow
    \item[2a)] Der Conveyor\_Belt ist nicht in Bewegung
    \begin{itemize}
        \item[2a1)] Starten des Conveyor_Belt
    \end{itemize}
    }

    \addCollum{Postcondition}{
    \item Der Conveyor\_Belt ist in Bewegung
    \item LightBarrier\_Start nicht mehr unterbrochen
    }
\end{usecase}

\begin{usecase}{11}{Run Service}{hoch}
    \addTextCollum{Verwandte Requirements}{
        REQ-11, REQ-15, REQ-40
    }
    \addTextCollum{Description}{
        Dieser Usecase betreibt das Durchführen des Servicemode. 
        Dieser kann durch das lange drücken des Starttasters von beiden Anlagen aus aktiviert werden.
        Im Servicemode werden die Höhenmesser kalibriert sowie die korrekte Funktion des Aussortiermechanismus geprüft
    }
    \addCollum{Actors}{
        \item Stoplight
        \item Height\_Measurement
        \item Control\_Panel
        \item Sorting\_Mechanism
    }
    \addCollum{Preconditions}{
        \item System ist im Ruhezustand
    }
    \addCollum{Triggers}{
        \item Start-Taster 3 Sekunden gedrückt
        \item Mitteilung von FTS\_2 über Betreten des Service
    }
    \addCollum{Mainflow}{
        \item[1)] Mitteilung an FTS\_2 über Betreten des Service senden
        \item[2)] Stoplight grün blinken lassen
        \item[3)] Messung bei Height\_Measurement anfragen
        \item[4)] Auf Rückmeldung warten
        \item[5)] Bestimmten Wert als Nullwert im Height\_Measurement eintragen
        \item[6)] Sorting Mechanism aktivieren
        \item[7)] Auf Zustandsänderungssignal des Sorting\_Mechanism warten \textbullet
        \item[8)] Sorting Mechanism deaktivieren
        \item[9)] Auf Zustandsänderungssignal des Sorting\_Mechanism warten \textbullet
        \item[10)] In Ruhezustand wechseln
    }
    \addCollum{Postconditions}{
        \item[1)] System ist im Ruhezustand
        \item[2)] Height\_Measurement wurde kalibriert
    }
    \addCollum{Exceptional flow 1}{
        \item at step 6 or 8 of Mainflow
        \item[1)] Timeout beim Empfang des Signals
        \item[2)] Fehler auslösen
    }
\end{usecase}


\begin{usecase}{04}{Transfer Workpiece}{mittel}
    \addTextCollum{Verwandte Requirements}{
    REQ-9, REQ-14, REQ-16, REQ-18, REQ-31
    }
    \addTextCollum{Description}{
    In diesem Use-Case erreicht ein Werkstück das Ende eines Förderbandes und wird gegebenenfalls an das nächste Förderband transferiert
    }
    \addTextCollum{Actors}{
    Conveyor\_Belt, FTS\_2, LightBarrier\_End
    }

    \addCollum{Precondition}{
    \item Anlage befindet sich im Betriebszustand
    }

    \addCollum{Trigger}{
    \item LightBarrier\_End wird unterbrochen
    }

    \addCollum{Mainflow}{
    \item[1)] Die Anlage ist in Primary Mode \textbullet
    \item[2)] Anfrage zum transferieren an FTS\_2 senden
    \item[3)] Auf 'OK' von FTS\_2 warten, Conveyor\_Belt ist solange nicht in Bewegung
    \item[4)] Conveyor\_Belt in Bewegung setzen und Werkstück transferieren
    }

    \addCollum{Alternative Flow 1}{
    \item at step 1) of Mainflow
    \item[1a)] Die Anlage ist in Secondary Mode
    \begin{itemize}
        \item[1a1)] Ausgabe der ID, des Typs, Höhe\_FB1 und Höhe\_FB2 auf der Konsole
        \item[1a2)] Ausgabe an der Konsole, ob sich das Werkstück bei der Übergabe zwischen den Anlagen überschlagen hat
    \end{itemize}
    }

    \addCollum{Postcondition}{
    \item Werkstück ist nicht mehr auf dem Förderband
    }
\end{usecase}

\begin{usecase}{9}{Run Service}{hoch}
    \addTextCollum{Verwandte Requirements}{
        REQ-11, REQ-15, REQ-40
    }
    \addTextCollum{Description}{
        Dieser Usecase betreibt das Durchführen des Servicemode. 
        Dieser kann durch das lange drücken des Starttasters von beiden Anlagen aus aktiviert werden.
        Im Servicemode werden die Höhenmesser kalibriert sowie die korrekte Funktion des Aussortiermechanismus geprüft
    }
    \addCollum{Actors}{
        \item Stoplight
        \item Height\_Measurement
        \item Control\_Panel
        \item Sorting\_Mechanism
    }
    \addCollum{Preconditions}{
        \item System ist im Ruhezustand
    }
    \addCollum{Triggers}{
        \item Start-taster wird lange gedrückt
        \item Mitteilung von FTS\_2 über Betreten des Service
    }
    \addCollum{Mainflow}{
        \item[1)] Mitteilung an FTS\_2 über Betreten des Service senden
        \item[2)] Stoplight grün blinken lassen
        \item[3)] Messung bei Height\_Measurement anfragen
        \item[4)] Auf Rückmeldung warten
        \item[5)] Bestimmten Wert als Nullwert im Height\_Measurement eintragen
        \item[6)] Sorting Mechanism aktivieren
        \item[7)] Auf Zustandsänderungssignal des Sorting\_Mechanism warten \textbullet
        \item[8)] Sorting Mechanism deaktivieren
        \item[9)] Auf Zustandsänderungssignal des Sorting\_Mechanism warten \textbullet
        \item[10)] Stoplight Grün leuchten lassen
    }
    \addCollum{Postconditions}{
        \item[1)] System ist im Ruhezustand
        \item[2)] Height\_Measurement wurde kalibriert
    }
    \addCollum{Exceptional flow 1}{
        \item at step 6 or 8 of Mainflow
        \item[1)] Timeout beim Empfang des Signals
        \item[2)] Fehler auslösen
    }
    \addCollum{Postconditions of Exceptional Flow 1}{
        \item[1)] System ist im Fehlerzustand
    }
\end{usecase}


\begin{usecase}{16}{Answer Transfer Request}{mittel}
    \addTextCollum{Verwandte Requirements}{
    REQ-9, REQ-14, REQ-16, REQ-18, REQ-31
    }
    \addTextCollum{Description}{
    In diesem Use-Case wird eine Transfer Request, die von der Master-Anlage ausgeht, von der Slave-Anlage beantwortet
    }
    \addTextCollum{Actors}{
    Conveyor\_Belt, FTS\_2, LightBarrier\_Height, LightBarrier\_Start
    }
    \addCollum{Precondition}{
    \item Anlage ist im Betriebszustand
    \item Anlage befindet sich im Secondary Mode
    }
    \addCollum{Trigger}{
    \item Transfer Request mit Workpiece-Info von FTS\_2
    }
    \addCollum{Mainflow}{
    \item[1)] Kein Werkstück befindet sich zwischen LightBarrier\_Start und LightBarrier\_Height \textbullet
    \item[2)] Sende 'OK' an FTS\_2
    }

    \addCollum{Alternative Flow 1}{
    \item at stept 1) of Mainflow
    \item[1a)] Es befindet sich ein Werkstück zwischen LightBarrier\_Start und LightBarrier\_Height
    \begin{itemize}
        \item[1a1)] Warten auf Unterbrechung von LightBarrier\_Height
        \item[1a1)] Go back to step 2) of Mainflow
    \end{itemize}
    }

    \addCollum{Postcondition}{
    \item Transfer Request beantwortet
    }
\end{usecase}

\begin{usecase}{02}{Determine Sorting Action}{hoch}
    \addTextCollum{Verwandte Requirements}{
        REQ-1, REQ-2, REQ-3, REQ-6
    }
    \addTextCollum{Description}{
        Es wird bestimmt, ob ein workpiece aussortiert werden soll, oder nicht.
        Dafür muss der Vorgänger workpiece\_type gespeichert werden.
    }
    \addTextCollum{Actors}{
        LightBarrier\_Ramp
    }
    \addCollum{Precondition}{
        \item Dem workpiece ist ein workpiece\_type zugeordnet
    }
    \addCollum{Main flow}{
        \item[1)] Der aktuelle workpiece\_type wird mit dem erwarteten verglichen
        \item[2)] Das workpiece stimmt nicht überein, das workpiece muss aussortiert werden \textbullet
        \item[3)] Das workpiece wird mit discard markiert \textbullet
    }
    \addCollum{Alternative flow 1}{
        \item at step 2 of main flow
        \item[2a)] Das workpiece stimmt überein
        \begin{itemize}
            \item[2a1)] Das workpiece wird mit do\_not\_discard markiert
            \item[2a2)] Ende des use case
        \end{itemize}
    }
    \addCollum{Alternative flow 2}{
        \item at step 3 of main flow
        \item[3a)] Rutsche ist voll und in Primary mode
        \begin{itemize}
            \item[3a1)] das Teil mit do\_not\_discard markiert
        \end{itemize}
        \item[3b)] Rutsche ist voll und in Secondary mode
        \begin{itemize}
            \item[3b1)] Gesamtbetrieb wird gestoppt (REQ-6)
        \end{itemize}
    }
    \addTextCollum{Postcondition}{
        Dem workpiece ist entweder do\_not\_discard oder discard zugeordnet
    }
\end{usecase}


\begin{usecase}{10}{Bootup Configuration}{hoch}
    \addTextCollum{Description}{
        Under construction
    }
\end{usecase}


\begin{usecase}{17}{Detect Material}{hoch}
    \addTextCollum{Verwandte Requirements}{
    REQ-1, REQ-2, REQ-3
    }
    \addTextCollum{Description}{
    In diesem Use-Case wird das Material eines Werkstückes erfasst.
    Grundsätzlich wird angenommen, dass ein Werkstück aus Kunststoff ist.
    }
    \addTextCollum{Actors}{
    Metal\_Detector
    }
    \addCollum{Precondition}{
    \item Die Anlage befindet sich im Betriebszustand
    }
    \addCollum{Trigger}{
    \item Metal\_Detector erfasst Metall
    }
    \addCollum{Mainflow}{
    \item[1)] Aktuellem Werkstück wird 'Metall' als Material zugewiesen
    }
    \addCollum{Postcondition}{
    \item Werkstück wurde Material 'Metall' zugewiesen
    }
\end{usecase}


<<<<<<< latex/Chapters/3_requirements.tex
\begin{figure}
    \centering
    \includegraphics[width=\textwidth]{../out/diagrams/stage1/req-fehlerzustand.pdf}
    \caption{Fehlerzustände}
    \label{fig:stm_fehler}
\end{figure}

=======
>>>>>>> latex/Chapters/3_requirements.tex
\section{Systemanalyse}\label{sec:systemanalyse}

%% Ihr technisches System hat aus Sicht der Software bestimmte Eigenschaften.
%% Was muss man für die Entwicklung der Software in Struktur, Schnittstellen,
%% Verhalten und an Besonderheiten wissen?
%% Wählen Sie eine Kapitelstruktur, die am besten zur Dokumentation Ihrer Ergebnisse geeignet ist.

\subsection{Art des Systems}

Bei dem zu entwerfenden System handelt es sich um die Steuerung eines Festo-Transfersystems, welche auf einem in die Anlage integrierten BeagleBoneBlack Einplatinencomputer zu realisieren ist.
Daher handelt es sich um ein Embedded System. 
Für das Zu entwickelnde Softwaresystem bedeutet dies, dass es zur Steuerung der Prozesse sehr nah an der in der Anlage verbauten Hardware arbeiten muss. 
Ein weiterer wichtiger Punkt ist, dass das System mit einer weiteren gleichartigen Anlage über das Netzwerk kommunizieren muss. Dies ist ebenfalls im Entwurfsprozess zu berücksichtigen.

\subsubsection{Eigenschaften des BeagleBoneBlack und Grundlagen QNX}

Auf dem BeagleBoneBlack Einplatinencomputer läuft das Echtzeitbetriebssystem QNX, das Kommunikation zwischen Prozessen und Threads mittels Message Passing bevorzugt.
Dieses Message Passing kann auch über das Netzwerk erfolgen, sodass Messages auch unkompliziert an einen weiteren BeagleBoneBlack geschickt werden können. 
Die Architektur sollte sich daher auf Message-Passing stützen, gerade um die Kommunikation mit der anderen gleichartigen Anlage zu erleichtern.
Da der BeagleBoneBlack ein eigenes Stück Hardware mit eigenem Betriebssystem ist, muss für die Entwicklung die Entwicklungsumgebung QNX Momentics benutzt werden, die Debugging und Testing ermöglicht.
Die Kommunikation mit dem BeagleBoneBlack und dem Entwicklungsrechner erfolgt über die Netzwerkverbindung.

\subsection{Kommunikation mit der Hardware}

Die Ansteuerung der Aktorik und Sensorik der Anlage erfolgt direkt über die GPIOs des BeagleBoneBlack. 
Die Sensorik ist hierbei an GPIO0, die Aktorik des Transfersystems an GPIO1 und die LEDs des Bedienpanels an GPIO2.
Bei den GPIOs können Bits einzeln gesetzt und gelöscht werden und es können die aktuellen Werte ausgelesen werden.
Auf Veränderung an den GPIOs kann mittels pro Pin konfigurierbarer Interrupts (steigende Flanke, fallende Flanke, Level=0, Level=1) reagiert werden.
Interrupts lassen sich nach dem Auslösen für einen einstellbare Zeit ignorieren (Debouncing).
Um später nicht direkt mit den GPIOs interagieren zu müssen, sollte ein Hardware Abstraction Layer einfachere Schnittstellen zum Ansteuern der Aktorik und für das Reagieren auf die Sensorik bereitstellen.
Der Höhensensor ist am ADC des BeagleBoneBlack angeschlossen. Dieser kann über das Schreiben in ein bestimmtes GPIO-Register aktiviert werden und kann dann nach erfolgreicher Messung einen Interrupt auslösen.

\subsection{Besonderheiten beim Systemaufbau}

Bei genauerer Betrachtung des Systemaufbaus fallen folgende Eigenschaften und Zusammenhänge besonders auf. 
Sie sind bei der Verhaltensmodellierung zu berücksichtigen und könnten das Erkennen von Szenarien vereinfachen.

\subsubsection{Platzierung der Lichtschranke bei der Höhenmessung}

Die Lichtschranke an der Höhenmessung ist so platziert, dass sich die Mitte eines Werkstücks bei ihrer Unterbrechung genau unter dem Höhenmesser befindet. 
Auf diese Art lassen sich die unterschiedlichen Formen von Werkstücken mittels einer einzelnen Messung Unterscheiden. 
Die gemessenen Höhen bei unterschiedlichen Werkstückarten gehen aus Tabelle~\ref{tab:werkstuecke} hervor.

\begin{table}[h]
    \begin{center}
        \begin{tabular}{ |c|c| }
            \hline
            Form                     & Höhe in mm \\
            \hline\hline
            HOCH                 &  25,0-25,4\\
            \hline
            FLACH                     & 21 \\
            \hline
            LOCH               & 15,8-16,4 \\
            \hline
        \end{tabular}
    \end{center}
    \caption{Höhen der unterschiedlichen Werkstückarten}
    \label{tab:werkstuecke}
\end{table}

Ebenso fällt auf, dass der Abstand der Lichtschranke für die Höhenmessung vom Anfang des Förderbands ca. 25 cm beträgt, was für die Bestimmung des Abstands bei der Werkstückübergabe zwischen zwei Anlagen wichtig sein könnte.

\subsubsection{Der Aufbau im Bereich des Aussortiermechnaismus}

Im Transportweg der Werkstücke befindet sich an der Stelle an der im Falle einer Weiche diese spätestens für Durchlass öffnen muss und im Falle eines Auswerfers dieser das Werkstück auswerfen muss eine Lichtschranke.
Das Unterbrechen dieser Lichtschranke kann daher als Signal zum Starten des Aussortiervorgangs verstanden werden. 
Oberhalb des Werkstücks befindet sich in dieser Position auch der Metallsensor.
Bei einer Anlage mit Weiche ist wichtig, dass diese bei längerem Verharren im geöffneten Zustand beschädigt werden kann.

\section{Softwareebene}\label{sec:softwareebene}

%% Sie sollen Software für die Steuerung des technischen Systems erstellen.
%% Aus den Anforderungen auf der Systemebene und der Systemanalyse ergeben sich
%% Anforderungen für Ihre Software.
%% Insbesondere wird sich die Software der beiden Anlagenteile in einigen Punkten unterscheiden.
%% Dokumentieren Sie hier die Anforderungen, die sich speziell für die Software ergeben haben.

\subsection{Systemkontext}\label{subsec:systemkontext}

%% Wie sieht der Kontext Ihrer Software aus? Wie erfolgt die Kommunikation mit Nachbarsystemen?
%% Liste der ein- und ausgehenden Signale/Nachrichten.

\subsection{Anforderungen}\label{subsec:anforderungen}

%% Welche wesentlichen Anforderungen ergeben sich aus den Systemanforderungen für Ihre Software?
%% Berücksichtigen Sie auch mögliche Fehlbedienungen und Fehlverhalten des Systems.
