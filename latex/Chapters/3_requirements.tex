\chapter{Requirements und Use Cases}\label{ch:requirements-und-use-cases}

\section{Systemebene}\label{sec:systemebene}

%% Die Anforderungen aus der Aufgabenstellung sind nicht vollständig. Die Struktur der nachfolgenden Kapitel soll Sie bei der Strukturierung der Analyse unterstützen. Dokumentieren Sie die Ergebnisse der Analysen entsprechend.

\subsection{Stakeholder}\label{subsec:stakeholder}

%% Ermitteln Sie die Stakeholder für das Projekt und listen Sie diese hier auf.

\paragraph{Externe Stakeholder}
\begin{itemize}
    \item Auftraggeber
    \begin{itemize}
        \item Erfüllung aller spezifizierten Anforderungen
        \item Pünktliche Lieferung zum vorgegebenen Termin
        \item Verwendung der vorgegebenen Hard- und Software
    \end{itemize}
    \item Betreuer
    \begin{itemize}
        \item Erreichen der Ziele am Ende jeder Phase
        \item Vollständige Dokumentation als Rückmeldungsgrundlage
    \end{itemize}
    \item Benutzer
    \begin{itemize}
        \item System stellt keine Gefahr dar
        \item Information über Fehlerzustände
        \item Seltenes Eingreifen erforderlich
        \item Hoher Durchsatz
        \item Einfache Bedienung und Inbetriebnahme (Dokumentation)
    \end{itemize}
    \item Verwaltung TI-Labor
    \begin{itemize}
        \item Keine Beschädigung der Anlagen
    \end{itemize}
\end{itemize}

\paragraph{Interne Stakeholder}
\begin{itemize}
    \item Entwickler
    \begin{itemize}
        \item Gute Testbarkeit
        \item Einfache Erweiterbarkeit und Modularität
        \item Einheitliche Schnittstellen und Benennungen
        \item Dokumentation (im Code) für Fehlersuche und Teamarbeit
    \end{itemize}
\end{itemize}

\paragraph{Akteure}

\begin{itemize}
    \item Control\_Panel\\
    \gls{ctrlp} der Anlage.
    Es stellt den \gls{t_start}, \gls{t_stop} und \gls{t_reset} zur Verfügung.

    \item Conveyor\_Belt\\
    Auf dem \gls{belt} werden Werkstücke durch das System bewegt.
    \item FTS\_2\\
    Die jeweils andere Anlage des Systems.
    \item E\_Stop\\
    Ein Button zum Auslösen des Emergency-Stops. Ist er gedrückt, kann er durch Herausziehen
    und drücken des \gls{t_reset} deaktiviert werden.
    \item Metal\_Detector\\
    Sensor zum Erfassen von Metall eines Werkstückes.
    \item Height\_Measurement\\
    Höhen-Sensor zum Erfassen der Höhe eines Werkstückes.
    \item Stoplight\\
    \gls{ampelled}, die durch verschiedene LED-Einstellungen Zustandsinformationen der Anlage anzeigt.
    \item Sorting\_Mechanism\\
    Jede Anlage besitzt einen Aussortier-Mechanismus.
    Dieser ist in einer von zwei Varianten eingebaut:
    \begin{itemize}
        \item Ejector\\
        Der \gls{ejector}, der bei Stromzufluss ausfährt und so Werkstücke in die Rampe stoßen soll.
        Ohne Strom steht der \gls{ejector} auf \gls{do_not_discard}.
        \item Switch\\
        Die \gls{weiche}, die bei Stromzufluss auf \gls{do_not_discard} steht und Werkstücke durchlässt.
        Ohne Strom steht die  \gls{weiche} auf \gls{discard} und befördert Werkstücke
        langsam in die Rampe.
    \end{itemize}
    \item LightBarrier\\
    Das System besitzt diverse Lichtschranken. Sie werden als Trigger und zur Steuerung
    des Kontrollflusses benutzt:
    \begin{itemize}
        \item \gls{lb_st}\\
        Lichtschranke am Anfang eines Förderbandes.
        Die Unterbrechung dieser Lichtschranke setzt, vorausgesetzt es liegen keine
        Fehler vor, das \gls{belt} in Bewegung.
        \item \gls{lb_ra}\\
        Lichtschranke an der Rampe. Aussortierte Werkstücke passieren diese Lichtschranke.
        Die kurze Unterbrechung dieser Lichtschranke kann als erfolgreiche Aussortierung interpretiert werden.
        Eine dauerhafte Unterbrechung signalisiert eine volle Rampe.
        \item \gls{lb_he}\\
        Lichtschranke an der Position des Höhensensors.
        Sie wird als Trigger für den Start der Höhenmessung benutzt.
        \item \gls{lb_sw}\\
        Lichtschranke vor dem Aussortier-Mechanismus der Anlage.
        Die Unterbrechung startet den Aussortieralgorithmus.
        \item \gls{lb_en}\\
        Lichtschranke am Ende eines Förderbandes.
        Die Unterbrechung dieser Lichtschranke kann das Transferieren des Werkstückes an die nächste Anlage einleiten.
        Am Ende der zweiten Anlage werden bei Unterbrechung dieser Lichtschranke Werkstück-Informationen an der Konsole ausgegeben.
    \end{itemize}
\end{itemize}

\subsection{Anforderungen}\label{subsec:anforderungen2}
Im Folgenden werden die Anforderungen an das System definiert. Diese wurden zur besseren
Übersichtlichkeit kategorisiert. Zum besseren Abgleich wird in den
Anforderungen jeweils auf die Sätze in den Anforderungen des Kunden verwiesen, aus denen diese hervorgehen.
Das wird, wie folgendes Beispiel zeigt, gekennzeichnet: (36). 

%! suppress = LabelConvention
\paragraph{\glspl{workpiece} und Sortierung}
\begin{itemize}
    \reqitem{1} Die \gls{anlage} kann zwischen vier Typen von \glspl{workpiece}n unterscheiden (\gls{workpiece_type}) (2, 3, 4, 5, 6)
    \begin{itemize}
        \item \gls{workpiece_flach} (3)
        \item \gls{workpiece_metall} (4)
        \item \gls{workpiece_bohrung} (5)
        \item \gls{workpiece_hoch} (6)
    \end{itemize}
    \reqitem{2} Am Ende des 2.\ \gls{belt}es sollen die \glspl{workpiece} zyklisch in folgender Reihenfolge ankommen (1, 7, 8)
    \begin{enumerate}
        \item \gls{workpiece_metall}
        \item \gls{workpiece_bohrung}
        \item \gls{workpiece_flach}
    \end{enumerate}
    \reqitem{3} \glspl{workpiece}, die nicht in die Sortierreihenfolge passen werden in eine der beiden \glspl{rampe} aussortiert (9, 10)
    \reqitem{4} Der \gls{durchsatz} an \glspl{workpiece} soll hoch sein (11)
    \reqitem{18} Falls sich bei der Übergabe zwischen den beiden \glspl{belt} ein \gls{workpiece}
    überschlägt, muss der neue \gls{workpiece_type} beachtet werden (siehe \refreq{47}) (20)
    \begin{itemize}
        \item Der Fall, dass das Teil auf die Seite fällt, sodass es wegrollen könnte, wird ausgeschlossen.
    \end{itemize}
    \reqitem{30} Aussortierung der \glspl{workpiece} soll mit \gls{weiche} funktionieren (41)
    \reqitem{38} Aussortierung der \glspl{workpiece} soll mit \gls{ejector} funktionieren (41)
    \reqitem{39} Beliebige Kombinationen der \gls{sortierer} an den beiden \glspl{anlage} sollen unterstützt werden (42)
    \reqitem{47} Wenn ein \gls{workpiece_bohrung} oder \gls{workpiece_metall} umgedreht wird, ist es ein \gls{workpiece_hoch}
\end{itemize}

\paragraph{Kapazität}
\begin{itemize}
    \reqitem{5} Bei einer vollen \gls{rampe} wird eine Warnung ausgesandt (12)
    \reqitem{6} Wenn die nächste notwendige Aussortierung aufgrund von ausgeschöpfter \glspl{rampe}kapazität
    nicht stattfinden kann, wird ein Fehler ausgesandt (und somit der Gesamtbetrieb gestoppt) (13)
\end{itemize}

\paragraph{Durchlassablauf}
\begin{itemize}
    \reqitem{7} Zuführung von \glspl{workpiece}n erfolgt durch Einlegen von \glspl{workpiece}n am Anfang von \gls{anlage} 1 (14, 15)
    \begin{itemize}
        \item Ein Unterbrechen der \gls{lb_st} signalisiert dem System das Einlegen eines \gls{workpiece}s,
        sodass der Transport dessen beginnen kann
    \end{itemize}
    \reqitem{9} Das System muss mit in beliebigem Abstand eingelegten \glspl{workpiece}n umgehen können (16, 17) %TODO Mindestabstand
    \begin{itemize}
        \item Solange der Bereich der ersten Lichtschranke frei ist, muss der Benutzer \glspl{workpiece}
        einlegen können, ohne die Korrektheit der Funktion zu gefährden
    \end{itemize}
    \reqitem{14} Der Abstand von \glspl{workpiece}n auf \gls{belt} 2 muss mindestens 25 cm betragen (18)
    \begin{itemize}
        \item Abstand muss vor der Übergabe sichergestellt werden
    \end{itemize}
    \reqitem{16} Auf dem \gls{belt} von \gls{anlage} 2 dürfen sich maximal 2 \glspl{workpiece} befinden (19)
    \reqitem{20} \Glspl{workpiece} dürfen nicht vom \gls{belt} fallen (21)
    \reqitem{24} Beim Einlegen eines \glspl{workpiece}s in die \gls{anlage} soll dem \gls{workpiece} eine eindeutige ID zugewiesen werden(28)
    \reqitem{26} Wenn sich auf einem \gls{belt} kein \gls{workpiece} befindet, stoppt das \gls{belt}
    \reqitem{31} Wenn ein \gls{workpiece} die \gls{lb_en} von \gls{anlage} 2 erreicht,
    sollen Informationen zu diesem \gls{workpiece} auf der Konsole ausgegeben werden (22, 23, 24, 25, 26, 27)
    \begin{itemize}
        \item Zu den Informationen zählen die ID, Typ, Höhe auf \gls{anlage} 1 und \gls{anlage} 2 des \gls{workpiece}es als
        auch ein Hinweis darüber, ob sich das \gls{workpiece} überschlagen hat
    \end{itemize}
\end{itemize}

\paragraph{Bedienung durch Taster}
\begin{itemize}
    \reqitem{12} Bei Betätigung von \gls{t_start} wechselt die \gls{anlage} in den Betriebszustand (49)
    \reqitem{15} Bei \gls{longpress} von \gls{t_start} wechselt die \gls{anlage} in den Service-Modus (50)
    \begin{itemize}
        \item Anforderung für den Wechsel ist, dass die \gls{anlage} im Ruhezustand ist
    \end{itemize}
    \reqitem{17} Bei Betätigung des \gls{t_stop} wechselt die \gls{anlage} in den Ruhezustand (51, 52)
    \begin{itemize}
        \item Wenn Fehler oder Warnung vorliegen, wird stattdessen ein Fehler ausgesandt  %TODO Warnung und Fehler in glossar aufnehmen
    \end{itemize}
    \reqitem{21} Bei Betätigung des \gls{t_reset} werden sämtliche Fehler quittiert (53) %TODO mit kunden kären
    \reqitem{28} Wenn die \gls{anlage} durch \gls{estop} stillgelegt ist, kann der Betrieb durch Drücken des
    \gls{t_reset} der \gls{anlage}, an dem auch der \gls{estop} gedrückt wurde, fortgesetzt werden (56) %TODO 'fortsetzten' mit kunden kären
    \begin{itemize}
        \item Bedingung dafür: Keine \gls{estop} sind gedrückt
    \end{itemize}
    \reqitem{40} Im Service Modus führt die \gls{anlage} Kalibrierung und Selbsttests durch (50) %TODO genauer spezifizieren %TODO Modi in glossar aufnehmen
    \reqitem{41} Bei Betätigung eines \gls{estop} werden beide \glspl{anlage} angehalten (54, 55)
    \reqitem{42} Dem Benutzer werden Hinweise über die Benutzung der \gls{anlage} mithilfe der LEDs an den \gls{taster}n gegeben
    \begin{itemize}
        \item Im Betriebszustand ist die LED am \gls{t_start} an
        \item Im Ruhezustand die LED am \gls{t_stop} an
        \item Bei einem gegangenen oder bestehenden Fehler ist die LED am \gls{t_reset} an
    \end{itemize}
\end{itemize}

\paragraph{Zustandsanzeigen}
\begin{itemize}
    \reqitem{10} Im Betriebszustand leuchtet die grüne \gls{ampelled} dauerhaft (59)
    \reqitem{11} Im Service-Mode blinkt die grüne \gls{ampelled}  (60)
    \reqitem{13} Bei Warnungen blinkt die gelbe \gls{ampelled} bei der \gls{anlage}, bei der die Warnung vorliegt (61)
    \reqitem{19} Wenn im Betriebszustand keine Warnungen vorliegen, ist die gelbe \gls{ampelled} aus (61)
    \reqitem{37} Die rote \gls{ampelled} signalisiert die Fehlerzustände wie folgt (73, 74, 75, 76):
    \begin{enumerate}
        \item\label{req-37-unq} Anstehend unquittiert wird durch schnelles Blinken (1 Hz) signalisiert (74)
        \item\label{req-37-quit} Anstehend quittiert wird durch dauerhaftes Leuchten(75) signalisiert
        \item\label{req-37-geg} Gegangen unquittiert wird durch langsames Blinken (0,5 Hz) signalisiert (z.B.\ wenn ein
        \gls{workpiece} an einer \gls{weiche} zu langsam in die \gls{rampe} geschoben wurde) (76)
        \item\label{req-37-ok} Steht kein Fehler an, ist die Leuchte aus (73)
    \end{enumerate}
    \reqitem{45} Im Ruhezustand leuchtet die \gls{ampel} dauerhaft gelb
\end{itemize}

\paragraph{\gls{weiche}}
\begin{itemize}
    \reqitem{23} Bei Verklemmen der \gls{weiche} wird eine Warnung ausgesandt, bis das \gls{workpiece} in der Rampe ankommt (37)
    \begin{itemize}
        \item Ein \gls{workpiece} ist verklemmt, wenn das \gls{workpiece} länger als erwartet braucht, um in der \gls{rampe} anzukommen
        \item Länger als erwartet wird mit mehr als 50 Prozent der durchschnittlichen Aussortierzeit definiert
    \end{itemize}
    \reqitem{27} Die \gls{weiche} darf nicht länger als 30 Sekunden auf \gls{do_not_discard} stehen (35, 36)
    \begin{itemize}
        \item Bei minutenlangen Stromfluss wird die \gls{weiche} beschädigt
    \end{itemize}
\end{itemize}

\paragraph{\gls{recorder}}
\begin{itemize}
    \reqitem{25} Es soll eine \gls{record-fn} bereitgestellt werden, mit der ein Benutzer alle
    \glspl{event} der \gls{anlage} in ein \gls{protokoll} aufzeichnen kann (90)
    \reqitem{29} Die von der \gls{record-fn} vorgenommene Aufzeichnung soll menschenlesbar sein (91)
    \reqitem{33} Es soll eine \gls{replay-fn} bereitgestellt werden, mit der ein
    Benutzer eine zuvor aufgezeichnetes \gls{protokoll} abspielen lassen kann (93, 94)
    \reqitem{34} \glspl{protokoll} sollen per Hand angefertigt werden können (95, 96)
\end{itemize}

\paragraph{Höhenmessung}
\begin{itemize}
    \reqitem{32} Bei der Auswertung der Höhenmessung ist die durch Verkippung des Sensors entstehende Abweichung zu berücksichtigen (43, 44)
\end{itemize}

\paragraph{Fehlerumgang}
\begin{itemize}
    \reqitem{35} Nach Behebung eines Fehlers soll der Normalbetrieb fortgesetzt werden (45, 46)
    \begin{itemize}
        \item Nach Möglichkeit sollen die \glspl{belt} nicht geräumt werden
    \end{itemize}
    \reqitem{43} In den Zuständen 'bestehend\_unquittiert' und 'bestehend\_quittiert' bleiben die
    \glspl{belt} beider \glspl{anlage} stehen und \glspl{weiche} werden auf \gls{discard} gestellt
    \begin{itemize}
        \item Die Fehleranzeige mittels der \gls{ampel} ist in \refreq{37} spezifiziert
    \end{itemize}
    \reqitem{36} Fehlerzustand soll wie in Abbildung~\ref{fig:stm-fehlerzustand} beschrieben sein. (65, 66, 67, 69, 70)
    \reqitem{46} Der Fehlerzustand beider \glspl{anlage} wird wie folgt festgelegt:
    \begin{itemize}
        \item Der aktuell bestehende Fehler mit der höchsten Fehlerstufe entspricht dem Fehlerzustand beider \glspl{anlage}.
        Die Fehlerstufen lauten wie folgt:
    \begin{enumerate}
        \item Anstehend unquittiert
        \item Anstehend quittiert
        \item Gegangen unquittiert
        \item OK
    \end{enumerate}
    \end{itemize}
\end{itemize}

\begin{figure}[h]
    \centering
    \includegraphics[scale=0.5]{../out/diagrams/stage1/req-fehlerzustand}
    \caption{REQ-36 Visualisierung des Fehlerzustandes eines einzelnen Fehlers}
    \label{fig:stm-fehlerzustand}
\end{figure}

\FloatBarrier

%% In der Aufgabenstellung sind Anforderungen an das System gestellt.
%% Arbeiten Sie diese hier auf und ergänzen Sie diese entsprechend der Absprachen mit dem Betreuer.
%% Achten Sie auf die entsprechende Atribuierung.
%% Berücksichtigen Sie auch mögliche Fehlbedienungen und Fehlverhalten des Systems.

\subsection{Systemkontext}\label{subsec:systemkontext2}

\begin{figure}
    \centering
    \includegraphics[width=\textwidth]{../out/diagrams/stage1/systemkontext.pdf}
    \caption{Systemkontext}
    \label{fig:systemkontext}
\end{figure}


Betrachte Abbildung \ref{fig:systemkontext}.
Die Systemsicht bringt umliegende Aktorik des Systems und System selber miteinander in Beziehung.

%% Use Cases werden aus einer bestimmten Sicht erstellt.
%% Dokumentieren Sie diese mittels Kontextdiagramm oder Use Case Diagramm.
%% Die Use Cases und Test Cases müssen zu der hier verwendeten Nomenklatur konsistent sein.

\subsection{Use Cases / User Stories}\label{subsec:use-cases-user-stories}

In Abbildung~\ref{fig:ucd} sind die Use Cases in einem Use Case Diagramm dargestellt.
Die Farbgebung dient lediglich zur Veranschaulichung.

\begin{figure}[h]
    \centering
    \makebox[\textwidth][c]{\includegraphics[width=1.2\textwidth]{../out/diagrams/stage1/ucd}}
    \caption{Use Case Diagramm}
    \label{fig:ucd}
\end{figure}

%% Dokumentieren Sie hier, welche Use Cases/ User Stories Sie auf der Systemebene implementieren müssen.
%% Die Test Cases sollen später zu den Use Cases/ User Stories konsistent sein.

%% < Hier kommt die genaue Beschreibung der Use.
%% Pro Anforderung eine Tabelle benutzen. Die Tabelle nach Belieben vervielfältigen. >


\begin{usecase}{02}{Determine Sorting Action}{hoch}
    \addTextCollum{Verwandte Requirements}{
        REQ-1, REQ-2, REQ-3, REQ-6
    }
    \addTextCollum{Description}{
        Es wird bestimmt, ob ein workpiece aussortiert werden soll, oder nicht.
        Dafür muss der Vorgänger workpiece\_type gespeichert werden.
    }
    \addTextCollum{Actors}{
        LightBarrier\_Ramp
    }
    \addCollum{Precondition}{
        \item Dem workpiece ist ein workpiece\_type zugeordnet
    }
    \addCollum{Main flow}{
        \item[1)] Der aktuelle workpiece\_type wird mit dem erwarteten verglichen
        \item[2)] Das workpiece stimmt nicht überein, das workpiece muss aussortiert werden \textbullet
        \item[3)] Das workpiece wird mit discard markiert \textbullet
    }
    \addCollum{Alternative flow 1}{
        \item at step 2 of main flow
        \item[2a)] Das workpiece stimmt überein
        \begin{itemize}
            \item[2a1)] Das workpiece wird mit do\_not\_discard markiert
            \item[2a2)] Ende des use case
        \end{itemize}
    }
    \addCollum{Alternative flow 2}{
        \item at step 3 of main flow
        \item[3a)] Rutsche ist voll und in Primary mode
        \begin{itemize}
            \item[3a1)] das Teil mit do\_not\_discard markiert
        \end{itemize}
        \item[3b)] Rutsche ist voll und in Secondary mode
        \begin{itemize}
            \item[3b1)] Gesamtbetrieb wird gestoppt (REQ-6)
        \end{itemize}
    }
    \addTextCollum{Postcondition}{
        Dem workpiece ist entweder do\_not\_discard oder discard zugeordnet
    }
\end{usecase}


\begin{usecase}{03}{Handle Lightbarrier\_Switch}{hoch}
    \addTextCollum{Description}{
        Bindeglied zwischen \nameref{uc:02}, \nameref{uc:06} und \nameref{uc:15},
    }
    \addTextCollum{Trigger}{
        Light\_Barrier\_Switch unterbrochen
    }
    \addCollum{Mainflow}{
        \item[1)] UC2 determine sorting action
        \item[2)] Sorting action ist discard\_workpiece \textbullet
        \item[3)] UC6 discard\_workpiece
    }
    \addCollum{Alternate flow 2}{
        \item at step 2) of Mainflow
        \item[2a)] Sorting action ist do\_not\_discard\_workpiece
        \begin{itemize}
            \item[2a1)] UC15 do\_not\_discard\_workpiece
            \item[2a1)] Ende des use cases
        \end{itemize}
    }
    % postconditions sind in den jeweiligen unter UCs gehandelt.
\end{usecase}

\begin{usecase}{04}{Transfer Workpiece}{mittel}
    \addTextCollum{Verwandte Requirements}{
    REQ-9, REQ-14, REQ-16, REQ-18, REQ-31
    }
    \addTextCollum{Description}{
    In diesem Use-Case erreicht ein Werkstück das Ende eines Förderbandes und wird gegebenenfalls an das nächste Förderband transferiert
    }
    \addTextCollum{Actors}{
    Conveyor\_Belt, FTS\_2, LightBarrier\_End
    }

    \addCollum{Precondition}{
    \item Anlage befindet sich im Betriebszustand
    }

    \addCollum{Trigger}{
    \item LightBarrier\_End wird unterbrochen
    }

    \addCollum{Mainflow}{
    \item[1)] Die Anlage ist in Primary Mode \textbullet
    \item[2)] Anfrage zum transferieren an FTS\_2 senden
    \item[3)] Auf 'OK' von FTS\_2 warten, Conveyor\_Belt ist solange nicht in Bewegung
    \item[4)] Conveyor\_Belt in Bewegung setzen und Werkstück transferieren
    }

    \addCollum{Alternative Flow 1}{
    \item at step 1) of Mainflow
    \item[1a)] Die Anlage ist in Secondary Mode
    \begin{itemize}
        \item[1a1)] Ausgabe der ID, des Typs, Höhe\_FB1 und Höhe\_FB2 auf der Konsole
        \item[1a2)] Ausgabe an der Konsole, ob sich das Werkstück bei der Übergabe zwischen den Anlagen überschlagen hat
    \end{itemize}
    }

    \addCollum{Postcondition}{
    \item Werkstück ist nicht mehr auf dem Förderband
    }
\end{usecase}

\begin{usecase}{05}{Receive Workpiece}{hoch}
    \addTextCollum{Verwandte Requirements}{
        REQ-7, REQ-24, REQ-26
    }
    \addTextCollum{Description}{
        In diesem Use-Case wird das Förderband einer Anlage gestartet, sobald sie ein Werkstück erreicht
    }
    \addTextCollum{Actors}{
        Conveyor\_Belt, LightBarrier\_Start
    }
    \addCollum{Precondition}{
    \item Die Anlage befindet sich im Betriebszustand
    }
    \addCollum{Trigger}{
    \item LightBarrier\_Start
    }
    \addCollum{Mainflow}{
    \item[1)] Zuweisung einer eindeutigen ID an das betroffene Werkstück
    \item[2)] Der Conveyor\_Belt ist in Bewegung
    }

    \addCollum{Alternative Flow 1}{
    \item at step 2) of Mainflow
    \item[2a)] Der Conveyor\_Belt ist nicht in Bewegung
    \begin{itemize}
        \item[2a1)] Starten des Conveyor_Belt
    \end{itemize}
    }

    \addCollum{Postcondition}{
    \item Der Conveyor\_Belt ist in Bewegung
    \item LightBarrier\_Start nicht mehr unterbrochen
    }
\end{usecase}

\begin{usecase}{06}{Discard workpiece}{hoch}
    \addTextCollum{Verwandte Requirements}{
        REQ-3, REQ-5, REQ-23, REQ-30, REQ-38, REQ-39
    }
    \addTextCollum{Description}{
    In diesem Use-Case bewegt der Aussortiertmechanismus der Anlage ein Werkstück vom Förderband in die Rutsche
    }
    \addTextCollum{Actors}{
    LightBarrier\_Switch, LightBarrier\_Ramp, SortingMechanism
    }
    \addCollum{Precondition}{
    \item Anlage befindet sich im Betriebszustand
    \item Das betroffene Werkstück ist als 'discard' markiert
    }
    \addCollum{Mainflow}{
    \item[1)] Die Kapazität wird überprüft
    \item[2)] Die Kapazität der Rutsche ist nicht ausgelastet \textbullet \textbullet
    \item[3)] Führe Aussortierung mit SortingMechanism durch
    \item[4)] Warte auf Unterbrechung von LightBarrier\_Ramp \textbullet
    \item[5)] Setze SortingMechanism zurück
    }

    \addCollum{Alternative Flow 1}{
    \item at step 2) of Mainflow
    \item[2a)] Die Kapazität der Rutsche ist ausgelastet und im Primary Modus
    \begin{itemize}
        \item[2a1)] UC01: Do not discard
    \end{itemize}
    }

    \addCollum{Alternative flow 2}{
    \item at step 4) of Mainflow
    \item[4a)] LightBarrier\_Ramp ist nach der 1,5-fachen durchschnittlichen Aussortierzeit nicht unterbrochen worden
    \begin{itemize}
        \item[4a1)] Warnung wird gesendet
        \item[4a2)] Warte auf Unterbrechung von LightBarrier\_Ramp
        \item[4a3)] Go back to step 5) of Mainflow
    \end{itemize}
    }

    \addCollum{Postcondition}{
    \item Werkstück liegt in der Rampe
    \item LightBarrier\_Switch nicht mehr unterbrochen
    }

    \addCollum{Exceptional flow}{
    \item at step 2) of Mainflow
    \item[2b)] Die Kapazität der Rutsche ist ausgelastet und im Secondary Modus
    \begin{itemize}
        \item[2b1)] Die Anlage sendet einen Fehler aus
    \end{itemize}

    }

    \addCollum{Postcondition of e.f.}{
    \item LightBarrier\_Ramp unterbrochen
    \item Die Anlage befindet sich im Fehlerzustand
    }
\end{usecase}

\begin{usecase}{07}{E-Stop}{hoch}
    \addCollum{Actors}{
    \item E\_Stop, FTS\_2, Stoplight, Control\_Panel, Sorting\_Mechanism, Conveyor\_Belt
    }
    \addCollum{Precondition}{}
    \addCollum{Triggers}{
        \item[1)] E\_Stop-Taster von FTS wird betätigt
        \item[1a)] E\_Stop-Schalter von FTS\_2 wird betätigt
    }
    \addCollum{Mainflow}{
        \item[1)] Der aktuelle Zustand sämtlicher Sensoren und Aktoren von FTS und FTS\_2 wird gespeichert
        \item[2)] Conveyor\_Belt, Sorting\_Mechanism werden von FTS und FTS\_2 abgeschaltet.
        \item[3)] Die rote LED am Stoplight an FTS und an FTS\_2 wird eingeschaltet.
        \item[4)] E\_Stop-Schalter von FTS herausziehen
        \item[5)]
        \item[2a)] or tihs
    }
    \addCollum{Alternate flow}{
        \item st step 4) of Mainflow
        \item[2a)] E\_Stop-Schalter von FTS\_2 wurde betätigt
        \begin{itemize}
            \item[2a)] E\_Stop-Schalter von FTS\_2 herausziehen
            \item[2b)] go to 5) of Mainflow
        \end{itemize}
    }
    \addTextCollum{Description}{
        Lorem ipsum dolor sit amet, consectetur adipiscing elit,
        sed do eiusmod tempor incididunt ut labore et dolore magna aliqua.
        Ut enim ad minim veniam, quis nostrud exercitation ullamco laboris
        nisi ut aliquip ex ea commodo consequat. Duis aute irure dolor in
        reprehenderit in voluptate velit esse cillum dolore eu fugiat nulla pariatur.
        Excepteur sint occaecat cupidatat non proident, sunt in culpa qui officia
        deserunt mollit anim id est laborum
    }
\end{usecase}

\begin{usecase}{08}{Switch To Idle}{hoch}
    \addTextCollum{Verwandte Requirements}{
        REQ-17,
    }
    \addTextCollum{Description}{
        Dieser Use-Case beschreibt, wie man in den Ruhezustand, bzw. idle wechselt.
    }
    \addCollum{Actors}{
        \item Sorting\_Mechanism
        \item Stoplight
        \item Control\_Panel
        \item FTS\_2
    }
    \addCollum{Precondition}{
        \item Es sind keine Fehler oder Warnungen vorhanden
        \item System ist im Betriebszustand
    }
    \addCollum{Triggers}{
        \item Stopp-Taster am Control\_Panel von FTS oder FTS\_2 wird betätigt
    }
    \addCollum{Mainflow}{
        \item[1)] Zustand sämtlicher Sensoren und Aktoren wird gespeichert
        \item[2)] Lampen an FTS und FTS\_2 wechseln auf gelb leuchtend
    }
    \addCollum{Postcondition}{
        \item System ist im Ruhezustand
    }
\end{usecase}

\begin{usecase}{09}{Start System}{hoch}
    \addTextCollum{Verwandte Requirements}{
        REQ-12, REQ-42
    }
    \addTextCollum{Actors}{
        FTS\_2, Control\_Panel
    }
    \addCollum{Preconditions}{
        \item System ist gestoppt
        \item System ist nicht im Fehlerzustand
    }
    \addCollum{Trigger}{
        \item Der Startknopf auf dem Steuerpanel wird gedrückt
        \item FTS\_2 schickt signal zum Starten
    }
    \addCollum{Mainflow}{
        \item[1)] System wechselt in den Betriebszustand \textbullet
        \item[2)] LED im Stopptaster wird ausgeschaltet
        \item[3)] LED im Starttaster wird eingeschaltet \textbullet
    }
    \addCollum{Alternate flow 1}{
        \item at step 1 of Mainflow
        \item[1)] Trigger ist Drücken des Knopfes
        \item[2)] Signal zum Starten an FTS\_2 schicken
        \item[3)] Auf Bestätigung von FTS\_2 warten \textbullet
        \item[4)] mit Mainflow Step 1 weitermachen
    }
    \addCollum{Alternate flow 2}{
        \item at step 3 of Mainflow
        \item[1)] Trigger ist Signal von FTS\_2 des Knopfes
        \item[2)] Signal Bestätigung an FTS\_2 schicken
    }
    \addCollum{Postconditions}{
        \item[2)] FTS\_2 ist im Betriebszustand
        \item[3)] System ist im Betriebszustand
    }
    \addCollum{Exceptional flow 1}{
        \item at step 3 of Alternate Flow 1
        \item[1)] Timeout
        \item[2)] In Fehlerzustand wechseln
    }
\end{usecase}


\begin{usecase}{10}{Bootup Configuration}{hoch}
    \addTextCollum{Description}{
        Under construction
    }
\end{usecase}


\begin{usecase}{11}{Run Service}{hoch}
    \addTextCollum{Verwandte Requirements}{
        REQ-11, REQ-15, REQ-40
    }
    \addTextCollum{Description}{
        Dieser Usecase betreibt das Durchführen des Servicemode. 
        Dieser kann durch das lange drücken des Starttasters von beiden Anlagen aus aktiviert werden.
        Im Servicemode werden die Höhenmesser kalibriert sowie die korrekte Funktion des Aussortiermechanismus geprüft
    }
    \addCollum{Actors}{
        \item Stoplight
        \item Height\_Measurement
        \item Control\_Panel
        \item Sorting\_Mechanism
    }
    \addCollum{Preconditions}{
        \item System ist im Ruhezustand
    }
    \addCollum{Triggers}{
        \item Start-Taster 3 Sekunden gedrückt
        \item Mitteilung von FTS\_2 über Betreten des Service
    }
    \addCollum{Mainflow}{
        \item[1)] Mitteilung an FTS\_2 über Betreten des Service senden
        \item[2)] Stoplight grün blinken lassen
        \item[3)] Messung bei Height\_Measurement anfragen
        \item[4)] Auf Rückmeldung warten
        \item[5)] Bestimmten Wert als Nullwert im Height\_Measurement eintragen
        \item[6)] Sorting Mechanism aktivieren
        \item[7)] Auf Zustandsänderungssignal des Sorting\_Mechanism warten \textbullet
        \item[8)] Sorting Mechanism deaktivieren
        \item[9)] Auf Zustandsänderungssignal des Sorting\_Mechanism warten \textbullet
        \item[10)] In Ruhezustand wechseln
    }
    \addCollum{Postconditions}{
        \item[1)] System ist im Ruhezustand
        \item[2)] Height\_Measurement wurde kalibriert
    }
    \addCollum{Exceptional flow 1}{
        \item at step 6 or 8 of Mainflow
        \item[1)] Timeout beim Empfang des Signals
        \item[2)] Fehler auslösen
    }
\end{usecase}


\begin{usecase}{12}{Handle Error}{hoch}
    \addTextCollum{Verwandte Requirements}{
    REQ-36, REQ-37
    }
    \addTextCollum{Description}{
    Dieser Use Case beschreibt die Fehlerzustände und Anzeige der roten LED beim Auftritt eines Fehlers
    Fehler müssen quittiert und behoben werden.
    }
    \addTextCollum{Actors}{
    Stoplight, Control\_Panel
    }\addTextCollum{Precondition}{
    Ein Fehler im System ist neu aufgetreten
    }
    \addTextCollum{Trigger}{
    Ein Fehler ist aufgetreten
    }
    \addCollum{Mainflow}{
    \item[1)] Das System geht in den Fehlerzustand astehend\_unqittiert
    \item[2)] Rote LED blinkt schnell(1Hz)
    \item[3)] Warte bis Reset-Taster gedrückt wird  \textbullet
    \item[4)] Fehler wurde acknowledged
    \item[5)] Fehlerzustand wechselt zu astehend\_qittiert
    \item[6)] Rote LED leuchtet dauerhaft
    \item[7)] Warte bis alle Fehler behoben sind
    \item[8)] System verlässt Fehlerzustand
    \item[9)] Rote LED wird ausgeschaltet
    }
    \addCollum{Alternate flow 1}{
    \item at step 3 of Mainflow
    \item[3a)] Fehler wurde behoben
    \begin{itemize}
        \item[3a1)] Fehlerzustand wechselt gegangen\_unqittiert
        \item[3a2)] Rote LED blinkt langsam(0,5Hz)
        \item[3a3)] Warte bis Reset-Taster gedrückt wird
        \item[3a4)] Fehler wurde acknowledged
        \item[3a5)] Zum Schritt 7 des main flow zurückkehren
    \end{itemize}
    }
    \addTextCollum{Postcondition}{
    Aufgetretener Fehler wurden behoben und System ist wieder im Normalbetrieb
    }
\end{usecase}

\begin{usecase}{14}{Measure Height}{hoch}
    \addTextCollum{Verwandte Requirements}{
        REQ-1, REQ-31, REQ-32
    }
    \addTextCollum{Description}{
        Die Höhe eines workpieces wird gemessen und das Höhenprofil dessen bestimmt.
    }
    \addTextCollum{Actors}{
        Height\_Measurement
    }
    \addCollum{Preconditions}{
        \item System ist im Betriebszustand
    }
    \addTextCollum{Trigger}{
        LightBarrier\_Height
    }
    \addCollum{Mainflow}{
        \item[1)] Messung bei Height\_Measurement anfragen
        \item[2)] Auf Rückmeldung warten
        \item[3)] Rückgabewert mit den in Höhenprofile erwähnten Werten vergleichen
        \item[4)] Rückgabewert fällt in einen der Bereiche (plus Toleranz) \textbullet
        \item[4)] Jeweiliges Höhenprofile zurückgeben
    }
    \addCollum{Exceptional Flow}{
        \item at item 4) of main flow
        \item[4)] Rückgabewert liegt außerhalb aller Bereiche
        \begin{itemize}
            \item Fehlermeldung ausgeben
        \end{itemize}
    }
    \addCollum{Additional Information}{
        \item Höhenprofile
        \begin{itemize}
            \item BOHRUNG 15,8 bis 16,4mm
            \item FLACH 21,0mm
            \item HOCH 25,0 bis 25,4mm
        \end{itemize}
    }
\end{usecase}


\begin{usecase}{15}{Do not discard}{hoch}
    \addTextCollum{Verwandte Requirements}{
        REQ-2, REQ-27
    }
    \addTextCollum{Description}{
        In diesem Use-Case wird ein Werkstück vom Aussortiermechanismus durchgelassen
    }
    \addTextCollum{Actors}{
        Sorting\_Mechanism, LightBarrier\_Switch
    }
    \addCollum{Precondition}{
        \item Die Anlage befindet sich im Betriebszustand
        \item Das betroffene Werkstück ist als \gls{do_not_discard} markiert
    }

    \addCollum{Mainflow}{
        \item[1)] Öffne Sorting\_Mechanism
        \item[2)] Setze Sorting\_Mechanism zurück, wenn LightBarrier\_Switch nicht mehr unterbrochen
    }

    \addCollum{Postcondition}{
        \item Sorting\_Mechanism zurückgesetzt
        \item Werkstück nicht aussortiert
    }
\end{usecase}

\begin{usecase}{16}{Answer Transfer Request}{mittel}
    \addTextCollum{Verwandte Requirements}{
    REQ-9, REQ-14, REQ-16, REQ-18, REQ-31
    }
    \addTextCollum{Description}{
    In diesem Use-Case wird eine Transfer Request, die von der Master-Anlage ausgeht, von der Slave-Anlage beantwortet
    }
    \addTextCollum{Actors}{
    Conveyor\_Belt, FTS\_2, LightBarrier\_Height, LightBarrier\_Start
    }
    \addCollum{Precondition}{
    \item Anlage ist im Betriebszustand
    \item Anlage befindet sich im Secondary Mode
    }
    \addCollum{Trigger}{
    \item Transfer Request mit Workpiece-Info von FTS\_2
    }
    \addCollum{Mainflow}{
    \item[1)] Kein Werkstück befindet sich zwischen LightBarrier\_Start und LightBarrier\_Height \textbullet
    \item[2)] Sende 'OK' an FTS\_2
    }

    \addCollum{Alternative Flow 1}{
    \item at stept 1) of Mainflow
    \item[1a)] Es befindet sich ein Werkstück zwischen LightBarrier\_Start und LightBarrier\_Height
    \begin{itemize}
        \item[1a1)] Warten auf Unterbrechung von LightBarrier\_Height
        \item[1a1)] Go back to step 2) of Mainflow
    \end{itemize}
    }

    \addCollum{Postcondition}{
    \item Transfer Request beantwortet
    }
\end{usecase}

\begin{usecase}{17}{Detect Material}{hoch}
    \addTextCollum{Verwandte Requirements}{
    REQ-1, REQ-2, REQ-3
    }
    \addTextCollum{Description}{
    In diesem Use-Case wird das Material eines Werkstückes erfasst.
    Grundsätzlich wird angenommen, dass ein Werkstück aus Kunststoff ist.
    }
    \addTextCollum{Actors}{
    Metal\_Detector
    }
    \addCollum{Precondition}{
    \item Die Anlage befindet sich im Betriebszustand
    }
    \addCollum{Trigger}{
    \item Metal\_Detector erfasst Metall
    }
    \addCollum{Mainflow}{
    \item[1)] Aktuellem Werkstück wird 'Metall' als Material zugewiesen
    }
    \addCollum{Postcondition}{
    \item Werkstück wurde Material 'Metall' zugewiesen
    }
\end{usecase}




\section{Systemanalyse}\label{sec:systemanalyse}

%% Ihr technisches System hat aus Sicht der Software bestimmte Eigenschaften.
%% Was muss man für die Entwicklung der Software in Struktur, Schnittstellen,
%% Verhalten und an Besonderheiten wissen?
%% Wählen Sie eine Kapitelstruktur, die am besten zur Dokumentation Ihrer Ergebnisse geeignet ist.

\subsection{Art des Systems}

Bei dem zu entwerfenden System handelt es sich um die Steuerung eines Festo-Transfersystems,
welche auf dem BeagleBoneBlack Einplatinencomputer, von dem jede Anlage einen besitzt, zu realisieren ist.
Daher handelt es sich um ein Embedded System.
Für das zu entwickelnde Softwaresystem bedeutet dies, dass es zur Steuerung der Prozesse sehr nah
an der in der Anlage verbauten Hardware arbeiten muss.
Ein weiterer wichtiger Punkt ist, dass das System mit einer weiteren gleichartigen Anlage über
das Netzwerk kommunizieren muss. Dies ist ebenfalls im Entwurfsprozess zu berücksichtigen.

\subsubsection{Eigenschaften des BeagleBoneBlack und Grundlagen QNX}

Auf dem BeagleBoneBlack Einplatinencomputer läuft das Echtzeitbetriebssystem QNX, das Kommunikation
zwischen Prozessen und Threads mittels Message Passing bevorzugt.
Dieses Message Passing kann auch über das Netzwerk erfolgen, sodass Messages auch an
einen weiteren BeagleBoneBlack geschickt werden können.
Die Architektur sollte sich daher auf Message-Passing stützen, gerade um die Kommunikation mit der
anderen gleichartigen Anlage zu erleichtern.
Da der BeagleBoneBlack ein eigenes Stück Hardware mit eigenem Betriebssystem ist, muss für die
Entwicklung die Entwicklungsumgebung QNX Momentics benutzt werden, die Debugging und Testing ermöglicht.
Die Kommunikation mit dem BeagleBoneBlack und dem Entwicklungsrechner erfolgt über die Netzwerkverbindung.

\subsection{Kommunikation mit der Hardware}

Die Ansteuerung der Aktorik und Sensorik der Anlage erfolgt direkt über die GPIOs des BeagleBoneBlack.
Die Sensorik ist hierbei an GPIO0, die Aktorik des Transfersystems an GPIO1 und die LEDs des Bedienpanels an GPIO2
gekoppelt.
Bei den GPIOs können Bits einzeln gesetzt und gelöscht werden und es können die aktuellen Werte ausgelesen werden.
Auf Veränderung an den GPIOs kann mittels pro Pin konfigurierbarer Interrupts (steigende Flanke, fallende Flanke,
Level=0, Level=1) reagiert werden.
Interrupts lassen sich nach dem Auslösen für einen einstellbare Zeit ignorieren (Debouncing).
Um später nicht direkt mit den GPIOs interagieren zu müssen, sollte ein Hardware Abstraction Layer
einfachere Schnittstellen zum Ansteuern der Aktorik und für das Reagieren auf die Sensorik bereitstellen.
Der Höhensensor ist am ADC des BeagleBoneBlack angeschlossen. Dieser kann über das Schreiben in ein
bestimmtes GPIO-Register aktiviert werden und kann dann nach erfolgreicher Messung einen Interrupt auslösen.

\subsection{Besonderheiten beim Systemaufbau}

Bei genauerer Betrachtung des Systemaufbaus fallen folgende Eigenschaften und Zusammenhänge besonders auf.
Sie sind bei der Verhaltensmodellierung zu berücksichtigen und könnten das Erkennen von Szenarien vereinfachen.

\subsubsection{Platzierung der Lichtschranke bei der Höhenmessung}

Die Lichtschranke an der Höhenmessung ist so platziert, dass sich die Mitte eines Werkstücks bei ihrer
Unterbrechung genau unter dem Höhenmesser befindet.
Auf diese Art lassen sich die unterschiedlichen Höhen von Werkstücken mittels einer einzelnen Messung unterscheiden.
Die gemessenen Höhen bei unterschiedlichen Werkstückarten gehen aus Tabelle~\ref{tab:werkstuecke} hervor.

\begin{table}[h]
    \begin{center}
        \begin{tabular}{ |c|c| }
            \hline
            Form                     & Höhe in mm \\
            \hline\hline
            HOCH                 &  25,0-25,4\\
            \hline
            FLACH                     & 19,8-20,2 \\
            \hline
            LOCH               & 15,8-16,4 \\
            \hline
        \end{tabular}
    \end{center}
    \caption{Höhen der unterschiedlichen Werkstückarten}
    \label{tab:werkstuecke}
\end{table}

Ebenso fällt auf, dass der Abstand der Lichtschranke für die Höhenmessung vom Anfang des Förderbands ca. 25 cm beträgt, 
was für die Bestimmung des Abstands bei der Werkstückübergabe zwischen zwei Anlagen wichtig ist.

\subsubsection{Der Aufbau im Bereich des Aussortiermechnaismus}

Im Transportweg der Werkstücke befindet sich an der Stelle, an der ein Werkstück vom Sortiermechanismus aussortiert
werden muss, eine Lichtschranke.
Das Unterbrechen dieser Lichtschranke kann daher als Signal zum Starten des Aussortiervorgangs verstanden werden.
Im Falle einer Weiche als Aussortiermechanismus müsste diese geöffnet werden, falls keine Aussortierung erfolgen soll.
Im Falle eines Auswerfers müsste dieser aktiviert werden, falls das Werkstück aussortiert werden muss.
Oberhalb des Werkstücks befindet sich in dieser Position auch der Metallsensor.
Bei einer Anlage mit Weiche ist wichtig, dass diese bei längerem Verharren im geöffneten Zustand beschädigt werden kann.

\section{Softwareebene}\label{sec:softwareebene}

%% Sie sollen Software für die Steuerung des technischen Systems erstellen.
%% Aus den Anforderungen auf der Systemebene und der Systemanalyse ergeben sich
%% Anforderungen für Ihre Software.
%% Insbesondere wird sich die Software der beiden Anlagenteile in einigen Punkten unterscheiden.
%% Dokumentieren Sie hier die Anforderungen, die sich speziell für die Software ergeben haben.

\subsection{Systemkontext}\label{subsec:systemkontext}

%% Wie sieht der Kontext Ihrer Software aus? Wie erfolgt die Kommunikation mit Nachbarsystemen?
%% Liste der ein- und ausgehenden Signale/Nachrichten.

\subsection{Anforderungen}\label{subsec:anforderungen}

%% Welche wesentlichen Anforderungen ergeben sich aus den Systemanforderungen für Ihre Software?
%% Berücksichtigen Sie auch mögliche Fehlbedienungen und Fehlverhalten des Systems.
