%! suppress = LabelConvention
\paragraph{\glspl{workpiece} und Sortierung}
\begin{itemize}
    \reqitem{1} Die \gls{anlage} kann zwischen vier Typen von \glspl{workpiece}n unterscheiden (\gls{workpiece_type}) (2, 3, 4, 5, 6)
    \begin{itemize}
        \item \gls{workpiece_flach} (3)
        \item \gls{workpiece_metall} (4)
        \item \gls{workpiece_bohrung} (5)
        \item \gls{workpiece_hoch} (6)
    \end{itemize}
    \reqitem{2} Am Ende des 2.\ \gls{belt}es sollen die \glspl{workpiece} zyklisch in folgender Reihenfolge ankommen (1, 7, 8)
    \begin{enumerate}
        \item \gls{workpiece_metall}
        \item \gls{workpiece_bohrung}
        \item \gls{workpiece_flach}
    \end{enumerate}
    \reqitem{3} \glspl{workpiece}, die nicht in die Sortierreihenfolge passen werden in eine der beiden \glspl{rampe} aussortiert (9, 10)
    \reqitem{4} Der \gls{durchsatz} an \glspl{workpiece} soll hoch sein (11)
    \reqitem{18} Falls sich bei der Übergabe zwischen den beiden \glspl{belt} ein \gls{workpiece}
    überschlägt, muss der neue \gls{workpiece_type} beachtet werden (siehe \refreq{47}) (20)
    \begin{itemize}
        \item Der Fall, dass das Teil auf die Seite fällt, sodass es wegrollen könnte, wird ausgeschlossen.
    \end{itemize}
    \reqitem{30} Aussortierung der \glspl{workpiece} soll mit \gls{weiche} funktionieren (41)
    \reqitem{38} Aussortierung der \glspl{workpiece} soll mit \gls{ejector} funktionieren (41)
    \reqitem{39} Beliebige Kombinationen der \gls{sortierer} an den beiden \glspl{anlage} sollen unterstützt werden (42)
    \reqitem{47} Wenn ein \gls{workpiece_bohrung} oder \gls{workpiece_metall} umgedreht wird, ist es ein \gls{workpiece_hoch}
\end{itemize}

\paragraph{Kapazität}
\begin{itemize}
    \reqitem{5} Bei einer vollen \gls{rampe} wird eine Warnung ausgesandt (12)
    \reqitem{6} Wenn die nächste notwendige Aussortierung aufgrund von ausgeschöpfter \glspl{rampe}kapazität
    nicht stattfinden kann, wird ein Fehler ausgesandt (und somit der Gesamtbetrieb gestoppt) (13)
\end{itemize}

\paragraph{Durchlassablauf}
\begin{itemize}
    \reqitem{7} Zuführung von \glspl{workpiece}n erfolgt durch Einlegen von \glspl{workpiece}n am Anfang von \gls{anlage} 1 (14, 15)
    \begin{itemize}
        \item Ein Unterbrechen der \gls{lb_st} signalisiert dem System das Einlegen eines \gls{workpiece}s,
        sodass der Transport dessen beginnen kann
    \end{itemize}
    \reqitem{9} Das System muss mit in beliebigem Abstand eingelegten \glspl{workpiece}n umgehen können (16, 17) %TODO Mindestabstand
    \begin{itemize}
        \item Solange der Bereich der ersten Lichtschranke frei ist, muss der Benutzer \glspl{workpiece}
        einlegen können, ohne die Korrektheit der Funktion zu gefährden
    \end{itemize}
    \reqitem{14} Der Abstand von \glspl{workpiece}n auf \gls{belt} 2 muss mindestens 25 cm betragen (18)
    \begin{itemize}
        \item Abstand muss vor der Übergabe sichergestellt werden
    \end{itemize}
    \reqitem{16} Auf dem \gls{belt} von \gls{anlage} 2 dürfen sich maximal 2 \glspl{workpiece} befinden (19)
    \reqitem{20} \Glspl{workpiece} dürfen nicht vom \gls{belt} fallen (21)
    \reqitem{24} Beim Einlegen eines \glspl{workpiece}s in die \gls{anlage} soll dem \gls{workpiece} eine eindeutige ID zugewiesen werden(28)
    \reqitem{26} Wenn sich auf einem \gls{belt} kein \gls{workpiece} befindet, stoppt das \gls{belt}
    \reqitem{31} Wenn ein \gls{workpiece} die \gls{lb_en} von \gls{anlage} 2 erreicht,
    sollen Informationen zu diesem \gls{workpiece} auf der Konsole ausgegeben werden (22, 23, 24, 25, 26, 27)
    \begin{itemize}
        \item Zu den Informationen zählen die ID, Typ, Höhe auf \gls{anlage} 1 und \gls{anlage} 2 des \gls{workpiece}es als
        auch ein Hinweis darüber, ob sich das \gls{workpiece} überschlagen hat
    \end{itemize}
\end{itemize}

\paragraph{Bedienung durch Taster}
\begin{itemize}
    \reqitem{12} Bei Betätigung von \gls{t_start} wechselt die \gls{anlage} in den Betriebszustand (49)
    \reqitem{15} Bei \gls{longpress} von \gls{t_start} wechselt die \gls{anlage} in den Service-Modus (50)
    \begin{itemize}
        \item Anforderung für den Wechsel ist, dass die \gls{anlage} im Ruhezustand ist
    \end{itemize}
    \reqitem{17} Bei Betätigung des \gls{t_stop} wechselt die \gls{anlage} in den Ruhezustand (51, 52)
    \begin{itemize}
        \item Wenn Fehler oder Warnung vorliegen, wird stattdessen ein Fehler ausgesandt  %TODO Warnung und Fehler in glossar aufnehmen
    \end{itemize}
    \reqitem{21} Bei Betätigung des \gls{t_reset} werden sämtliche Fehler quittiert (53) %TODO mit kunden kären
    \reqitem{28} Wenn die \gls{anlage} durch \gls{estop} stillgelegt ist, kann der Betrieb durch Drücken des
    \gls{t_reset} der \gls{anlage}, an dem auch der \gls{estop} gedrückt wurde, fortgesetzt werden (56) %TODO 'fortsetzten' mit kunden kären
    \begin{itemize}
        \item Bedingung dafür: Keine \gls{estop} sind gedrückt
    \end{itemize}
    \reqitem{40} Im Service Modus führt die \gls{anlage} Kalibrierung und Selbsttests durch (50) %TODO genauer spezifizieren %TODO Modi in glossar aufnehmen
    \reqitem{41} Bei Betätigung eines \gls{estop} werden beide \glspl{anlage} angehalten (54, 55)
    \reqitem{42} Dem Benutzer werden Hinweise über die Benutzung der \gls{anlage} mithilfe der LEDs an den \gls{taster}n gegeben
    \begin{itemize}
        \item Im Betriebszustand ist die LED am \gls{t_start} an
        \item Im Ruhezustand die LED am \gls{t_stop} an
        \item Bei einem gegangenen oder bestehenden Fehler ist die LED am \gls{t_reset} an
    \end{itemize}
\end{itemize}

\paragraph{Zustandsanzeigen}
\begin{itemize}
    \reqitem{10} Im Betriebszustand leuchtet die grüne \gls{ampelled} dauerhaft (59)
    \reqitem{11} Im Service-Mode blinkt die grüne \gls{ampelled}  (60)
    \reqitem{13} Bei Warnungen blinkt die gelbe \gls{ampelled} bei der \gls{anlage}, bei der die Warnung vorliegt (61)
    \reqitem{19} Wenn im Betriebszustand keine Warnungen vorliegen, ist die gelbe \gls{ampelled} aus (61)
    \reqitem{37} Die rote \gls{ampelled} signalisiert die Fehlerzustände wie folgt (73, 74, 75, 76):
    \begin{enumerate}
        \item\label{req-37-unq} Anstehend unquittiert wird durch schnelles Blinken (1 Hz) signalisiert (74)
        \item\label{req-37-quit} Anstehend quittiert wird durch dauerhaftes Leuchten(75) signalisiert
        \item\label{req-37-geg} Gegangen unquittiert wird durch langsames Blinken (0,5 Hz) signalisiert (z.B.\ wenn ein
        \gls{workpiece} an einer \gls{weiche} zu langsam in die \gls{rampe} geschoben wurde) (76)
        \item\label{req-37-ok} Steht kein Fehler an, ist die Leuchte aus (73)
    \end{enumerate}
    \reqitem{45} Im Ruhezustand leuchtet die \gls{ampel} dauerhaft gelb
\end{itemize}

\paragraph{\gls{weiche}}
\begin{itemize}
    \reqitem{23} Bei Verklemmen der \gls{weiche} wird eine Warnung ausgesandt, bis das \gls{workpiece} in der Rampe ankommt (37)
    \begin{itemize}
        \item Ein \gls{workpiece} ist verklemmt, wenn das \gls{workpiece} länger als erwartet braucht, um in der \gls{rampe} anzukommen
        \item Länger als erwartet wird mit mehr als 50 Prozent der durchschnittlichen Aussortierzeit definiert
    \end{itemize}
    \reqitem{27} Die \gls{weiche} darf nicht länger als 30 Sekunden auf \gls{do_not_discard} stehen (35, 36)
    \begin{itemize}
        \item Bei minutenlangen Stromfluss wird die \gls{weiche} beschädigt
    \end{itemize}
\end{itemize}

\paragraph{\gls{recorder}}
\begin{itemize}
    \reqitem{25} Es soll eine \gls{record-fn} bereitgestellt werden, mit der ein Benutzer alle
    \glspl{event} der \gls{anlage} in ein \gls{protokoll} aufzeichnen kann (90)
    \reqitem{29} Die von der \gls{record-fn} vorgenommene Aufzeichnung soll menschenlesbar sein (91)
    \reqitem{33} Es soll eine \gls{replay-fn} bereitgestellt werden, mit der ein
    Benutzer eine zuvor aufgezeichnetes \gls{protokoll} abspielen lassen kann (93, 94)
    \reqitem{34} \glspl{protokoll} sollen per Hand angefertigt werden können (95, 96)
\end{itemize}

\paragraph{Höhenmessung}
\begin{itemize}
    \reqitem{32} Bei der Auswertung der Höhenmessung ist die durch Verkippung des Sensors entstehende Abweichung zu berücksichtigen (43, 44)
\end{itemize}

\paragraph{Fehlerumgang}
\begin{itemize}
    \reqitem{35} Nach Behebung eines Fehlers soll der Normalbetrieb fortgesetzt werden (45, 46)
    \begin{itemize}
        \item Nach Möglichkeit sollen die \glspl{belt} nicht geräumt werden
    \end{itemize}
    \reqitem{43} In den Zuständen 'bestehend\_unquittiert' und 'bestehend\_quittiert' bleiben die
    \glspl{belt} beider \glspl{anlage} stehen und \glspl{weiche} werden auf \gls{discard} gestellt
    \begin{itemize}
        \item Die Fehleranzeige mittels der \gls{ampel} ist in \refreq{37} spezifiziert
    \end{itemize}
    \reqitem{36} Fehlerzustand soll wie in Abbildung~\ref{fig:stm-fehlerzustand} beschrieben sein. (65, 66, 67, 69, 70)
    \reqitem{46} Der Fehlerzustand beider \glspl{anlage} wird wie folgt festgelegt:
    \begin{itemize}
        \item Der aktuell bestehende Fehler mit der höchsten Fehlerstufe entspricht dem Fehlerzustand beider \glspl{anlage}.
        Die Fehlerstufen lauten wie folgt:
    \begin{enumerate}
        \item Anstehend unquittiert
        \item Anstehend quittiert
        \item Gegangen unquittiert
        \item OK
    \end{enumerate}
    \end{itemize}
\end{itemize}

\begin{figure}[h]
    \centering
    \includegraphics[scale=0.5]{../out/diagrams/stage1/req-fehlerzustand}
    \caption{REQ-36 Visualisierung des Fehlerzustandes eines einzelnen Fehlers}
    \label{fig:stm-fehlerzustand}
\end{figure}
