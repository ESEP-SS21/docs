\begin{usecase}{13}{Durchführung Selbsttests}{mittel}
    \addTextCollum{Verwandte Requirements}{
        \refreq{40}, \refreq{15}, \refreq{11}
    }
    \addTextCollum{Description}{
        Dieser Usecase betreibt das Durchführen des Servicemode.
        Dieser kann durch das \gls{longpress} des \gls{t_start} von beiden \glspl{anlage} aus aktiviert werden.
        Im Servicemode werden die Höhenmesser kalibriert sowie die korrekte Funktion der \gls{sortierer} geprüft.
        Die detaillierte Durchführung wird im Kapitel "Verhaltensmodellierung" beschrieben.
    }
    \addTextCollum{Actors}{
        \gls{ampel}, \gls{ctrlp}, \gls{fts2}, \gls{belt}, \gls{sortierer}, \gls{t_start}, \gls{t_reset}
    }
    \addCollum{Precondition}{
        \item \gls{system} ist im \gls{ruhe-zst}
    }
    \addCollum{Triggers}{
        \item[1a)] \gls{t_start} \(\geq 3\) sek gedrückt 
    }
    \addCollum{Mainflow}{
        \item[2)] \gls{ampel} auf schnell grün blinken schalten
        \item[3)] \gls{he_sensor} kalibrieren
        \item[4)] \gls{belt} schnell vorwärts für 1 sek
        \item[5)] \gls{belt} schnell Rückwärts für 1 sek
        \item[6)] \gls{belt} stopp
        \item[7)] LED in \gls{t_reset} aktivieren
        \item[8)] Warten auf \gls{t_reset} gedrückt
        \item[9)] LED in \gls{t_reset} deaktivieren] 
        \item[10)] \gls{sortierer} aktivieren
        \item[11)] \gls{sortierer} deaktivieren
        \item[12)] LED in \gls{t_reset} aktivieren
        \item[13)] Warten auf \gls{t_reset} gedrückt
        \item[14)] LED in \gls{t_reset} deaktivieren
        \item[15)] \gls{ampel} auf gelb leuchten schalten
        \item[16)] Wechsel in Ruhezustand
    }
    \addCollum{Postcondition}{
        \item \gls{system} ist im \gls{ruhe-zst}
        \item \gls{he_sensor} ist kalibriert
    }
    \addCollum{Excepional Flow}{
        \item[1)] \gls{t_start} \(\geq 3\) sek gedrückt
        \item[2)] \gls{ampel} auf gelb leuchten schalten
        \item[3)] Wechsel in Ruhezustand
    }
\end{usecase}