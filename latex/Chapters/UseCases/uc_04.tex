\begin{usecase}{04}{\gls{workpiece} sortieren}{hoch}
    \addTextCollum{Verwandte Requirements}{
        \refreq{1}, \refreq{2}, \refreq{3}, \refreq{6}
    }
    \addTextCollum{Description}{
        Es wird bestimmt, ob ein \gls{workpiece} aussortiert werden soll, oder nicht.
        Dafür muss der Vorgänger \gls{workpiece_type} gespeichert werden.
    }
    \addTextCollum{Actors}{
        \gls{lb_sw}, \gls{lb_ra}
    }
    \addCollum{Precondition}{
        \item Dem \gls{workpiece} ist ein \gls{workpiece_type} zugeordnet
    }
    \addCollum{Trigger}{
        \item \gls{lb_sw} unterbrochen
    }
    \addCollum{Main flow}{
        \item[1)] Der aktuelle \gls{workpiece_type} wird mit dem erwarteten verglichen
        \item[2)] Das \gls{workpiece} stimmt nicht überein \textbullet
        \item[3)] Das \gls{workpiece} wird aussortiert % TODO UC Referenz
    }
    \addCollum{Alternative flow 1}{
        \item at step 2 of main flow
        \item[2a)] Das \gls{workpiece} stimmt überein
        \begin{itemize}
            \item[2a1)] Das \gls{workpiece} wird durchgelassen % TODO UC Referenz
            \item[2a2)] Ende des use case
        \end{itemize}
    }
    \addCollum{Alternative flow 2}{
        \item at step 3 of main flow
        \item[3a)] \gls{rampe} ist voll und in \gls{primary}
        \begin{itemize}
            \item[3a1)] das \gls{workpiece} wird durchgelassen
        \end{itemize}
        \item[3b)] \gls{rampe} ist voll und in \gls{secondary}
        \begin{itemize}
            \item[3b1)] Fehler wird gesendet
        \end{itemize}
    }
    \addCollum{Postcondition}{
        \item Das \gls{workpiece} ist sortiert
    }
\end{usecase}
