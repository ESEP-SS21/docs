\begin{usecase}{10}{Bootup Configuration}{hoch}
    \addTextCollum{Description}{
        Dieser Use Case beschreibt die Initialisierung des Systems, sowie die Aushandlung ob das System selbst oder das FTS\_2 das Primary System ist.
        Das Primary System wird durch Drücken des Startknopfes der entsprechenden Anlage bestimmt
    }
    \addTextCollum{Actors}{
         FTS\_2, Control\_Panel
    }
    \addCollum{Preconditions}{
        \item System ist mit Strom versorgt und Hochgefahren
    }
    \addCollum{Mainflow}{
        \item[1)] HAL wird initialisiert \textbullet
        \item[2)] Herstellen einer Verbindung zu FTS\_2 \textbullet
        \item[3)] Kontrolleuchte 1 auf Control\_Panel Blink\_Slow aktivieren
        \item[4)] Warten auf Drücken des Startknopfes auf dem ControlPanel \textbullet
        \item[5)] Primary-Anfrage an FTS\_2 schicken
        \item[6)] Warten auf Bestätigung von FTS\_2 \textbullet
        \item[7)] Eigenen Betriebsmodus auf PRIMARY setzen
        \item[8)] Kontrolleuchte 1 auf Control\_Panel dauerhaft aktivieren
    }
    \addCollum{Alternate flow 1}{
        \item at step 2 of Mainflow
        \item[1)] Verbindung schlägt fehl
        \item[2)] Zurück in Schritt 2 Mainflow
    }
    \addCollum{Alternate flow 2}{
        \item at step 4 of Mainflow
        \item[1)] Event Primary-Anfrage von FTS\_2 Empfangen
        \item[2)] Eigenen Betriebsmodus auf SECONDARY setzen
        \item[3)] Bestätigung an FTS\_2 senden
        \item[4)] Kontrolleuchte 2 auf Control\_Panel dauerhaft aktivieren
    }
    \addCollum{Postconditions}{
        \item[1)] HAL ist initialisiert
        \item[2)] Verbindung mit FTS\_2 hergestellt
        \item[3)] Betriebsmodus der Anlage ist festgelegt
    }
    \addCollum{Exceptional flow 1}{
        \item at step 1 of Mainflow
        \item[1)] Initialisierung der HAL schlägt fehl
        \item[2)] Fehlermeldung ausgeben
        \item[3)] System abschalten
    }
    \addCollum{Exceptional flow 2}{
        \item at step 6 of Mainflow
        \item[1)] Timeout beim Empfangen
        \item[2)] Fehlermeldung ausgeben
        \item[3)] System abschalten
    }
\end{usecase}
