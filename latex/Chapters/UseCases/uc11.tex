\begin{usecase}{11}{Run Service}{hoch}
    \addTextCollum{Verwandte Requirements}{
        REQ-11, REQ-15, REQ-40
    }
    \addTextCollum{Description}{
        Dieser Usecase betreibt das Durchführen des Servicemode.
        Dieser kann durch das lange drücken des Starttasters von beiden Anlagen aus aktiviert werden.
        Im Servicemode werden die Höhenmesser kalibriert sowie die korrekte Funktion des Aussortiermechanismus geprüft
    }
    \addCollum{Actors}{
        \item Stoplight
        \item Height\_Measurement
        \item Control\_Panel
        \item Sorting\_Mechanism
    }
    \addCollum{Preconditions}{
        \item System ist im Ruhezustand
    }
    \addCollum{Triggers}{
        \item Start-Taster 3 Sekunden gedrückt
        \item Mitteilung von FTS\_2 über Betreten des Service
    }
    \addCollum{Mainflow}{
        \item[1)] Mitteilung an FTS\_2 über Betreten des Service senden
        \item[2)] Stoplight grün blinken lassen
        \item[3)] Messung bei Height\_Measurement anfragen
        \item[4)] Auf Rückmeldung warten
        \item[5)] Bestimmten Wert als Nullwert im Height\_Measurement eintragen
        \item[6)] Sorting Mechanism aktivieren
        \item[7)] Auf Zustandsänderungssignal des Sorting\_Mechanism warten \textbullet
        \item[8)] Sorting Mechanism deaktivieren
        \item[9)] Auf Zustandsänderungssignal des Sorting\_Mechanism warten \textbullet
        \item[10)] In Ruhezustand wechseln
    }
    \addCollum{Postconditions}{
        \item[1)] System ist im Ruhezustand
        \item[2)] Height\_Measurement wurde kalibriert
    }
    \addCollum{Exceptional flow 1}{
        \item at step 6 or 8 of Mainflow
        \item[1)] Timeout beim Empfang des Signals
        \item[2)] Fehler auslösen
    }
\end{usecase}
