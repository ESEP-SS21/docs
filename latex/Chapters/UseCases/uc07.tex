\begin{usecase}{07}{E-Stop}{hoch}
    \addTextCollum{Verwandte Requirements}{
        REQ-41, REQ-42, REQ-28, TODO
    }
    \addTextCollum{Description}{
        Dieser Use-Case beschreibt den Umgang des Systems bei Betätigung des E\_Stop-Schalters.
    }
    \addCollum{Actors}{
        \item E\_Stop
        \item FTS\_2
        \item Stoplight
        \item Control\_Panel
        \item Sorting\_Mechanism
        \item Conveyor\_Belt
    }
    \addCollum{Precondition}{
        \item[1a)] Das System befindet sich im Ruhezustand
        \item[1b)] Das System befindet sich im Betriebszustand
    }
    \addCollum{Triggers}{
        \item[1a)] E\_Stop-Schalter von FTS wird betätigt
        \item[1b)] E\_Stop-Schalter von FTS\_2 wird betätigt
    }
    \addCollum{Mainflow}{
        \item[1)] Der aktuelle Zustand sämtlicher Sensoren und Aktoren von FTS und FTS\_2 wird gespeichert
        \item[2)] Conveyor\_Belt, Sorting\_Mechanism von FTS und FTS\_2 werden abgeschaltet.
        \item[3)] Lampe am FTS und FTS\_2 schaltet um auf rot blinkend.
        \item[4)] E\_Stop-Schalter von FTS wird herausgezogen
        \item[5)] Reset-Taster von FTS wird gedrückt
        \item[6)] Switch\_To\_Idle
    }
    \addCollum{Alternate flow}{
        \item at step 4) of Mainflow
        \item[2a)] E\_Stop-Schalter von FTS\_2 wurde betätigt (Trigger 1b))
        \begin{itemize}
            \item[1)] E\_Stop-Schalter von FTS\_2 wird herausgezogen
            \item[2)] Reset-Taster von FTS\_2 wird gedrueckt
            \item[3)] weiter mit 6) von Mainflow
        \end{itemize}
    }
    \addCollum{Postcondition}{
        \item Das System befindet sich im Ruhezustand.
    }
\end{usecase}