\begin{usecase}{02}{Determine Sorting Action}{hoch}
    \addTextCollum{Verwandte Requirements}{
        REQ-1, REQ-2, REQ-3, REQ-6
    }
    \addTextCollum{Description}{
        Es wird bestimmt, ob ein \gls{workpiece} aussortiert werden soll, oder nicht.
        Dafür muss der Vorgänger \gls{workpiece_type} gespeichert werden.
    }
    \addTextCollum{Actors}{
        LightBarrier\_Ramp
    }
    \addCollum{Precondition}{
        \item Dem workpiece ist ein \gls{workpiece_type} zugeordnet
    }
    \addCollum{Main flow}{
        \item[1)] Der aktuelle \gls{workpiece_type} wird mit dem erwarteten verglichen
        \item[2)] Das \gls{workpiece} stimmt nicht überein, das workpiece muss aussortiert werden \textbullet
        \item[3)] Das \gls{workpiece} wird mit \gls{discard} markiert \textbullet
    }
    \addCollum{Alternative flow 1}{
        \item at step 2 of main flow
        \item[2a)] Das \gls{workpiece} stimmt überein
        \begin{itemize}
            \item[2a1)] Das \gls{workpiece} wird mit \gls{do_not_discard} markiert
            \item[2a2)] Ende des use case
        \end{itemize}
    }
    \addCollum{Alternative flow 2}{
        \item at step 3 of main flow
        \item[3a)] Rutsche ist voll und in Primary mode
        \begin{itemize}
            \item[3a1)] das Teil mit \gls{do_not_discard} markiert
        \end{itemize}
        \item[3b)] Rutsche ist voll und in Secondary mode
        \begin{itemize}
            \item[3b1)] Gesamtbetrieb wird gestoppt (REQ-6)
        \end{itemize}
    }
    \addTextCollum{Postcondition}{
        Dem \gls{workpiece} ist entweder \gls{do_not_discard} oder \gls{discard} zugeordnet
    }
\end{usecase}
