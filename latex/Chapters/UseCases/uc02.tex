\begin{usecase}{02}{Determine Sorting Action}{hoch}
    \addCollum{Precondition}{
        \item Dem workpice ist ein workpice\_type zugeordnet
    }
    \addCollum{Actors}{
        \item LightBarrier\_Ramp
    }
    \addTextCollum{Description}{
        Es wird bestimmt, ob ein workpice aussortiert werden soll, oder nicht.
        Dafür muss der Vorgänger workpiece\_type gespeichert werden.
    }
    \addCollum{Main flow}{
        \item[1)] Der aktuelle workpiece\_type wird mit dem letzten verglichen.
        Wenn der aktuelle mit dem nach REQ-2 zyklischen Nachfolger des vorherigen übereinstimmt,
        wird das workpieces mit
        do\_not\_discard% <- TODO namen festlegen
        markiert und der use case endet hier.
        \item[2)] Das workpiece muss aussortiert werden. Es wird überprüft, ob die Rampe voll
        ist, indem LightBarrier\_Ramp überprüft wird. % <- das funktioniert noch nicht so ganz
        Wenn die Rampe voll ist, folgt alternative flow 2.
        \item[3)] Das workpice wird mit discard markiert.
    }
    \addCollum{Alternative flow 1}{
        \item[1a)] Wenn es noch keinen Vorgänger gibt, und das erste Teil der Liste vorkommt
        wird das workpieces mit
        do\_not\_discard % <- TODO namen festlegen
        markiert und der use case endet hier, ansonsten folgt 2).
    }
    \addCollum{Alternative flow 2}{
        \item[2a)] Befindet sich in Primary mode
        \begin{itemize}
            \item[2a1)] Wenn FB2 Rampe voll wird der Gesamtbetrieb gestoppt (REQ-6),
            sonst wird das Teil mit do\_not\_discard % <- TODO namen festlegen
            markiert.
        \end{itemize}
        \item[2b)] Befindet sich in Secondary mode
        \begin{itemize}
            \item[2b1)] Gesamtbetrieb wird gestoppt (REQ-6)
        \end{itemize}
    %TODO wie soll das genau funktionieren
    }

    \addTextCollum{Verwandte Requirements}{
        REQ-1, REQ-2, REQ-3, REQ-6
    }
\end{usecase}