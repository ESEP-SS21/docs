\begin{usecase}{02}{Determine Sorting Action}{hoch}
    \addCollum{Precondition}{
        \item Dem workpiece ist ein workpiece\_type zugeordnet
    }
    \addCollum{Actors}{
        \item LightBarrier\_Ramp
    }
    \addTextCollum{Description}{
        Es wird bestimmt, ob ein workpiece aussortiert werden soll, oder nicht.
        Dafür muss der Vorgänger workpiece\_type gespeichert werden.
    }
    \addCollum{Main flow}{
        \item[1)] Der aktuelle workpiece\_type wird mit dem erwarteten verglichen
        \item[2)] Das workpiece stimmt nicht überein, das workpiece muss aussortiert werden \textbullet
        \item[3)] Das workpiece wird mit discard markiert \textbullet
    }
    \addCollum{Alternative flow 1}{
        \item at step 2 of main flow
        \item[2a)] Das workpiece stimmt überein
        \begin{itemize}
            \item[2a1)] Das workpiece wird mit do\_not\_discard markiert
            \item[2a2)] Ende des use case
        \end{itemize}

%        Es wird überprüft, ob die Rampe voll ist, indem LightBarrier\_Ramp überprüft wird. % <- das funktioniert noch nicht so ganz
%        Wenn die Rampe voll ist, folgt alternative flow 2.
    }
    \addCollum{Alternative flow 2}{
        \item at step 3 of main flow
        \item[3a)] Rutsche ist voll und in Primary mode
        \begin{itemize}
            \item[3a1)]
            das Teil mit do\_not\_discard % <- TODO namen festlegen
            markiert
        \end{itemize}
        \item[3b)] Rutsche ist voll und in Secondary mode
        \begin{itemize}
            \item[3b1)] Gesamtbetrieb wird gestoppt (REQ-6)
        \end{itemize}
    %TODO wie soll das genau funktionieren
    }

    \addTextCollum{Verwandte Requirements}{
        REQ-1, REQ-2, REQ-3, REQ-6
    }
\end{usecase}