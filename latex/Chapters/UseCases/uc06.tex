\begin{usecase}{06}{Discard workpiece}{hoch}
    \addTextCollum{Verwandte Requirements}{
        REQ-3, REQ-5, REQ-23, REQ-30, REQ-38, REQ-39
    }
    \addTextCollum{Description}{
    In diesem Use-Case bewegt der Aussortiertmechanismus der Anlage ein Werkstück vom Förderband in die Rutsche
    }
    \addTextCollum{Actors}{
    LightBarrier\_Switch, LightBarrier\_Ramp, SortingMechanism
    }
    \addCollum{Precondition}{
    \item Anlage befindet sich im Betriebszustand
    \item Das betroffene Werkstück ist als 'discard' markiert
    }
    \addCollum{Mainflow}{
    \item[1)] Die Kapazität wird überprüft
    \item[2)] Die Kapazität der Rutsche ist nicht ausgelastet, LightBarrier\_Ramp also nicht unterbrochen\textbullet
    \item[3)] Führe Aussortierung mit SortingMechanism durch
    \item[4)] Warte auf Unterbrechung von LightBarrier\_Ramp
    \item[5)] LightBarrier\_Ramp wurde ausgelöst
    }

    \addCollum{Alternative Flow 1}{
    \item at step 2) of Mainflow
    \item[1)] Die Kapazität der Rutsche ist ausgelastet, LightBarrier\_Ramp also unterbrochen
    \item[2)] Das Förderband ist in Primary Mode\textbullet
    \item[3)] UC01: Do not discard
    }

    \addCollum{Postcondition}{
    \item Werkstück liegt in der Rampe
    \item LightBarrier\_Switch nicht mehr unterbrochen
    }

    \addCollum{Exceptional flow 1}{
        \item at step 5) of Mainflow
        \item[1)] LightBarrier\_Ramp ist nach der 1,5-fachen durchschnittlichen Aussortierzeit nicht ausgelöst worden
        \item[2)] Warnung wird gesendet
    }
    \addCollum{Postcondition of e.f.1}{
        \item LightBarrier\_Switch nicht mehr unterbrochen
        \item Eine Warnung über das Feststecken des Werkstückes liegt vor
    }

    \addCollum{Exceptional flow 2}{
    \item at step 2) of Alternative Flow 1
    \item[1)] Das Förderband ist in Secondary Mode
    \item[2)] Die Anlage geht im Fehlerzustand
    }

    \addCollum{Postcondition of e.f.2}{
    \item LightBarrier\_Ramp unterbrochen
    \item Die Anlage befindet sich im Fehlerzustand
    }
\end{usecase}