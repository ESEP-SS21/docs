\begin{usecase}{12}{Handle Error}{hoch}
    \addTextCollum{Actors}{
    Stoplight
    }\addTextCollum{Description}{
    Dieser Use Case beschreibt den Umgang mit Fehlern, die während des betriebes auftauchen können.
    Wenn Fehler auftreten, dann wird damit in diesem Use Case umgegangen.
    Fehler müssen quittiert und behoben werden.
    }\addTextCollum{Precondition}{
    Ein Fehler im System ist neu aufgetreten
    }
    \addCollum{Mainflow}{
    \item[1)] Das System geht in den Fehlerzustand astehend\_unqittiert
    \item[2)] Rote LED blinkt schnell(1Hz)
    \item[3)] UC13 Acknowledge Error \textbullet
    \item[4)] Fehlerzustand wechselt zu astehend\_qittiert
    \item[5)] Rote LED leuchtet dauerhaft
    \item[6)] Warte bis Fehler behoben
    \item[7)] System verlässt Fehlerzustand
    \item[8)] Rote LED wird ausgeschaltet
    }
    \addCollum{Alternate flow 1}{
    \item at step 3 of Mainflow
    \item[3a)] Fehler wurde behoben
    \begin{itemize}
        \item[3a1)] Fehlerzustand wechselt gegangen\_unqittiert
        \item[3a2)] Rote LED blinkt langsam(0,5Hz)
        \item[3a3)] UC13 Acknowledge Error.
        \item[3a4)] Zum Schritt 7 des main flow zurückkehren
    \end{itemize}
    }
    \addCollum{Alternate flow 2}{
    \item at any step of Use Case
    \item[1)] Ein weiterer Fehler tritt auf
    \item[2)] Zum Schritt 1 des main flow zurückkehren
    }
    \addTextCollum{Postcondition}{
    Aufgetretene Fehler wurden behoben und System ist wieder im Normalbetrieb
    }
\end{usecase}