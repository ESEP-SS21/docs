\begin{usecase}{12}{Handle Error}{hoch}
    \addTextCollum{Actors}{
    Stoplight
    }\addTextCollum{Description}{
    Dieser Use Case beschreibt den Umgang mit Fehlern, die während des betriebes auftauchen können.
    Wenn ein Fehler auftritt, dann wechselt das System in den Fehlerzustand.
    Dieser muss quittiert und behoben werden, dann wird der Zustand wieder verlassen
    }\addTextCollum{Precondition}{
    Ein Fehler im System ist neu aufgetreten
    }
    \addCollum{Mainflow}{
    \item[1)] Das System geht in den Fehlerzustand astehend\_unqittiert
    \item[2)] Rote LED blinkt schnell(1Hz)
    \item[3)] UC13 Acknowledge Error
    \item[4)] Fehlerzustand wechselt zu astehend\_qittiert
    \item[5)] Rote LED leuchtet dauerhaft
    \item[6)] Fehler wird behoben
    \item[7)] System verlässt Fehlerzustand
    \item[8)] Rote LED wird ausgeschaltet

    }\addCollum{Alternate flow}{
    \item at step 3 of Mainflow
    \item[3a)] Fehler repariert sich selbst
    \begin{itemize}
        \item[3a1)] Fehlerzustand wechselt gegangen\_unqittiert
        \item[3a2)] Rote LED blinkt langsam(0,5Hz)
        \item[3a3)] UC13 Acknowledge Error.
        \item[3a4)] Zum Schritt 7 des main flow zurückkehren
    \end{itemize}
    }\addTextCollum{Postcondition}{
    Fehlerzustand behoben und System wieder im Normalbetrieb
    }
\end{usecase}