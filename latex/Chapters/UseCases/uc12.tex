\begin{usecase}{12}{Handle Error}{hoch}
    \addTextCollum{Verwandte Requirements}{
    REQ-36, REQ-37
    }
    \addTextCollum{Description}{
    Dieser Use Case beschreibt die Fehlerzustände und Anzeige der roten LED beim Auftritt eines Fehlers
    Fehler müssen quittiert und behoben werden.
    }
    \addTextCollum{Actors}{
    Stoplight, Control\_Panel
    }\addTextCollum{Precondition}{
    Ein Fehler im System ist neu aufgetreten
    }
    \addTextCollum{Trigger}{
    Ein Fehler ist aufgetreten
    }
    \addCollum{Mainflow}{
    \item[1)] Das System geht in den Fehlerzustand astehend\_unqittiert
    \item[2)] Rote LED blinkt schnell(1Hz)
    \item[3)] Warte bis Reset-Taster gedrückt wird  \textbullet
    \item[4)] Fehler wurde acknowledged
    \item[5)] Fehlerzustand wechselt zu astehend\_qittiert
    \item[6)] Rote LED leuchtet dauerhaft
    \item[7)] Warte bis alle Fehler behoben sind
    \item[8)] System verlässt Fehlerzustand
    \item[9)] Rote LED wird ausgeschaltet
    }
    \addCollum{Alternate flow 1}{
    \item at step 3 of Mainflow
    \item[3a)] Fehler wurde behoben
    \begin{itemize}
        \item[3a1)] Fehlerzustand wechselt gegangen\_unqittiert
        \item[3a2)] Rote LED blinkt langsam(0,5Hz)
        \item[3a3)] Warte bis Reset-Taster gedrückt wird
        \item[3a4)] Fehler wurde acknowledged
        \item[3a5)] Zum Schritt 7 des main flow zurückkehren
    \end{itemize}
    }
    \addTextCollum{Postcondition}{
    Aufgetretener Fehler wurden behoben und System ist wieder im Normalbetrieb
    }
\end{usecase}