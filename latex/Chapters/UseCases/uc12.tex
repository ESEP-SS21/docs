\begin{usecase}{12}{Handle Error}{hoch}
    \addTextCollum{Verwandte Requirements}{
    REQ-21, REQ-36, REQ-37, REQ-43, REQ-46
    }
    \addTextCollum{Actors}{
    Stoplight, Control\_Panel
    }
    \addTextCollum{Description}{
    Der ErrorHandler verwaltet alle Fehler, die im System auftreten. Immer wenn ein Fehler-Event ausgelöst wird,
    legt der ErrorHandler diese in eine Fehlerliste ab. Jeder Fehler hat einen Fehlerzustand,
    welcher wie in der Statemachine~\ref{fig:stm_fehler} beschrieben ist.
    Wenn der Reset-Taster gedrückt wird, werden alle Fehler in der Fehlerliste quittiert.
    Fehler gelten als behoben, wenn ein Event die Behebung des Fehlers meldet.
    Der Fehlerzustand der gesamten Anlage, somit auch die Anzeige der roten LED,
    wird durch den dem Fehler mit der höchsten Priorität in der Fehlerliste festgelegt. Die Fehlerzustände sind wie in Fehlerpriorität priorisiert.
    Ist der globale Fehlerzustand anstehend\_unquittiert oder anstehend\_quittiert,
    dann bleiben die Laufbänder beider Anlagen stehen und der Strom an den Aussortiermechanismen wird abgestellt.
    }
    \addCollum{Fehlerpriorität}{
    \item[1)] anstehend\_unquittiert. Wird durch ein dauerhaftes leuchten der roten LED signalisiert.
    \item[2)] anstehend\_quittiert. Wird durch wird durch schnelles Blinken (1 Hz) der roten LED signalisiert.
    \item[3)] gegangen\_quittiert . Wird durch wird durch langsames Blinken (0,5 Hz) der roten LED signalisiert.
    }
\end{usecase}