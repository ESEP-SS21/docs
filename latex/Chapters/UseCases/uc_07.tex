\begin{usecase}{07}{\gls{workpiece}-Übergabe}{mittel}
    \addTextCollum{Verwandte Requirements}{
        \refreq{9}, \refreq{14}, \refreq{16}, \refreq{18}, \refreq{31}
    }
    \addTextCollum{Description}{
        In diesem Use-Case erreicht ein Werkstück das Ende des Förderbandes.
        Befindet sich das System in Primary Mode wird das \gls{workpiece} an die Anlage in Secondary Mode übergeben.
    }
    \addTextCollum{Actors}{
        Conveyor\_Belt, FTS\_2, LightBarrier\_End
    }

    \addCollum{Precondition}{
        \item Anlage befindet sich im Betriebszustand
    }

    \addCollum{Trigger}{
        \item LightBarrier\_End wird unterbrochen
    }

    \addCollum{Mainflow}{
        \item[1)] Die Anlage ist in Primary Mode \textbullet
        \item[2)] Conveyor\_Belt stoppen
        \item[3)] Anfrage zum transferieren und Werkstück\-Informationen an FTS\_2 senden
        \item[4)] Auf 'OK' von FTS\_2 warten
        \item[5)] Conveyor\_Belt in Bewegung setzen und Werkstück transferieren
    }

    \addCollum{Alternative Flow 1}{
        \item at step 1) of Mainflow
        \item[1a)] Die Anlage ist in Secondary Mode
        \begin{itemize}
            \item[1a1)] Ausgabe der ID, des Typs, Höhe\_FB1 und Höhe\_FB2 auf der Konsole
            \item[1a2)] Ausgabe an der Konsole, ob sich das Werkstück bei der Übergabe zwischen den Anlagen überschlagen hat
        \end{itemize}
    }

    \addCollum{Postcondition}{
        \item Werkstück ist nicht mehr auf dem Förderband der eigenen Anlage
    }
\end{usecase}