\begin{usecase}{08}{Switch To Idle}{mittel}
    \addTextCollum{Verwandte Requirements}{
        REQ-17, REQ-40
    }
    \addTextCollum{Description}{
        Dieser Use-Case beschreibt, wie vom Betriebszustand in den Ruhezustand, bzw. idle
        gewechselt wird.
    }
    \addTextCollum{Actors}{
        Sorting\_Mechanism, Stoplight, Control\_Panel, FTS\_2, Conveyor\_Belt
    }
    \addCollum{Precondition}{
        \item Es sind keine Fehler oder Warnungen vorhanden
        \item System ist im Betriebszustand
    }
    \addCollum{Triggers}{
        \item[1a)] Stopp-Taster am Control\_Panel betätigt
        \item[1b)] Event Request\_Idle von FTS\_2
    }
    \addCollum{Mainflow}{
        \item[1)] Zustand sämtlicher Sensoren und Aktoren wird gespeichert \textbullet
        \item[2)] Lampen an FTS und FTS\_2 wechseln auf gelb leuchtend
        \item[3)] LED am Start-Taster vom Control\_Panel ausschalten
        \item[4)] LED am Stopp-Taster vom Control\_Panel einschalten \textbullet
    }
    \addCollum{Alternative FLow 1}{
        \item at step 1 of Mainflow
        \item[1a)] Trigger war 1a)
        \begin{itemize}
            \item[2a1)] Request\_Idle wird an FTS\_2 gesendet
            \item[2a2)] Auf Acknowledge von FTS\_2 warten \textbullet
            \item[2a3)] mit step 1 of Mainflow weitermachen
        \end{itemize}
    }
    \addCollum{Alternative Flow 2}{
        \item at step 4 of Mainflow
        \item[4a)] Trigger war 1b)
        \begin{itemize}
            \item[4a1)] Acknowledge wird an FTS\_2 gesendet
        \end{itemize}
    }
    \addCollum{Postcondition}{
        \item System ist im Ruhezustand
    }
    \addCollum{Exceptional Flow 1}{
        \item at Step 2a2 of Alternative Flow 1
        \item[2a)] Timeout beim Warten auf Acknowledge
        \begin{itemize}
            \item[2a1)] Fehler auslösen
        \end{itemize}
    }
    \addCollum{Postcondition of e.f.}{
        \item System ist im Fehlerzustand
    }
\end{usecase}