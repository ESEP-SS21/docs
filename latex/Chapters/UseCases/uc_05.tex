\begin{usecase}{05}{\gls{workpiece} aussortieren}{hoch}
    \addTextCollum{Verwandte Requirements}{
    \refreq{3}, \refreq{5}, \refreq{23}, \refreq{30}, \refreq{38}, \refreq{39}
    }
    \addTextCollum{Description}{
        In diesem Use-Case bewegt der Aussortiermechanismus der Anlage ein Werkstück vom Förderband in die Rutsche
    }
    \addTextCollum{Actors}{
        LightBarrier\_Ramp, Sorting\_mechanism
    }
    \addCollum{Precondition}{
        \item Anlage befindet sich im Betriebszustand
    }
    \addCollum{Mainflow}{
        \item[1)] Führe Aussortierung mit sorting\_mechanism durch
        \item[2)] Warte auf Unterbrechung von LightBarrier\_Ramp \textbullet
        \item[3)] Setze sorting\_mechanism zurück
    }

    \addCollum{Alternative flow 1}{
        \item at step 4) of Mainflow
        \item[4a)] LightBarrier\_Ramp ist nach zu langer Aussortierzeit nicht unterbrochen worden, \refreq{23}
        \begin{itemize}
            \item[4a1)] Warnung wird gesendet
            \item[4a2)] Warte auf Unterbrechung von LightBarrier\_Ramp
            \item[4a3)] Sende Nachricht, dass Warnung behoben ist
            \item[4a4)] Go back to step 3) of Mainflow
        \end{itemize}
    }

    \addCollum{Postcondition}{
        \item Werkstück liegt in der Rampe
        \item sorting\_mechanism ist zurückgesetzt
    }
\end{usecase}