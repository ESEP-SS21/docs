\newglossary{term}{term}{term}{Begriffe}

\newglossaryentry{anlage}{
    type = {term},
    name = {Anlage},
    plural={Anlagen},
    description = {Eine der beiden Festo Transfersysteme inkl.\ aller Aktoren und Sensoren}
}

\newglossaryentry{durchsatz}{
    type = {term},
    name = {Durchsatz},
    description = {Behandelte \glspl{workpiece} pro Zeit}
}

\newglossaryentry{workpiece}{
    type = {term},
    name = {Werkstück},
    description = {Ein Teil, das auf dem \gls{belt} befördert wird},
    plural={Werkstücke}
}

\newglossaryentry{rampe}{
    type = {term},
    name = {Rampe},
    plural={Rampen},
    description = {Auffangbereich für \glspl{workpiece}},
}

\newglossaryentry{event}{
    type = {term},
    name = {Event},
    plural={Events},
    description = {Auftretendes Ereignis, dass unter den Komponenten mittels
    Dispatcher hin- und hergeschickt werden kann},
}

\newglossaryentry{protokoll}{
    type = {term},
    name = {Eventprotokoll},
    plural={Eventprotokolle},
    description = {Eine Aufzeichnung von \glspl{event}},
}

\newglossaryentry{replay-fn}{
    type = {term},
    name = {Replay-Funktion},
    description = {Funktion zum Abspielen der Aufzeichnung der Events},
}

\newglossaryentry{record-fn}{
    type = {term},
    name = {Record-Funktion},
    description = {Funktion zum Aufzeichnen der Events},
}

\newglossaryentry{primary}{
    type = {term},
    name = {Primary Mode},
    description = {Modus in dem die vordere \gls{anlage} befindet, an der die \glspl{workpiece} durch den Benutzer eingelegt werden. },
}

\newglossaryentry{secondary}{
    type = {term},
    name = {Secondary Mode},
    description = {Modus in dem die hintere \gls{anlage} befindet, an der die \glspl{workpiece} durch die andere \gls{anlage} übergeben werden. },
}
\newglossaryentry{betrieb-zst}{
    type = {term},
    name = {Betriebszustand},
    description = {Normalzustand des Systems; Umfasst auch Ruhe- und Fehlerzustand},
}
\newglossaryentry{ruhe-zst}{
    type = {term},
    name = {Ruhezustand},
    description = {Betriebszustand, in dem beide Förderbänder still stehen, der aber nicht durch
    einen Fehler hervorgerufen wurde},
}
\newglossaryentry{fehler-zst}{
    type = {term},
    name = {Fehlerzustand},
    description = {Betriebszustand nach Auslösen eines Fehlers},
}
\newglossaryentry{service-mode}{
    type = {term},
    name = {Service-Modus},
    description = {Modus, in dem das System Selbsttests und Kalibrierungen durchführt},
}
\newglossaryentry{system}{
    type = {term},
    name = {System},
    description = {Steuerung einer \gls{anlage}},
}
\newglossaryentry{recCrea}{ type = {term},
    name = {EmbeddedRecordCreator},
    description = {Programm zum Anzeigen, Bearbeiten und Erstellen von Eventabfolgen},
}
\newglossaryentry{embRecer}{ type = {term},
    name = {Embedded Recorder},
    description = {Modul zum Aufzeichnen und Abspielen von Eventabfolgen},
}
