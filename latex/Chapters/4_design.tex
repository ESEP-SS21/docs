\chapter{Design}\label{ch:design}

%% Anmerkung: Die Implementierung MUSS zu Ihrem Design-Modell konsistent sein.
%% Strukturen, Verhalten und Bezeichner im Code müssen mit dem Modell übereinstimmen.
%% Daher ist ein wohlüberlegtes Design wichtig.


\section{Systemarchitektur}\label{sec:systemarchitektur}

%% Erstellen Sie eine Architektur für Ihre Software.
%% Geben Sie eine kurze Beschreibung Ihrer Architektur mit den dazugehörenden Komponenten
%% und Schnittstellen an.
%% Dokumentieren Sie hier wichtige technische Entscheidungen.
%% Welche Pattern werden gegebenenfalls verwendet? Wie erfolgt die interne Kommunikation?

In Abbildung~\ref{fig:cmp} ist die Systemarchitektur mithilfe eines UML Komponentendiagramms
visualisiert.

\begin{figure}[h]
    \centering
    \makebox[\textwidth][c]{\includegraphics[width=1.2\textwidth]{../out/diagrams/stage1/cmp}}
    \caption{Komponentendiagramm}
    \label{fig:cmp}
\end{figure}

\begin{figure}
    \makebox[\textwidth][c]{\includegraphics[width=1.25\textwidth]{../out/diagrams/stage2/cd_hal}}
    \caption{Klassendiagramm HAL}
    \label{fig:cd-hal}
\end{figure}

\begin{figure}
    \makebox[\textwidth][c]{\includegraphics[width=1.25\textwidth]{../out/diagrams/stage3/cd_hal_sens}}
    \caption{}
    \label{fig:cd-hal-sens}
\end{figure}

\begin{figure}
    \makebox[\textwidth][c]{\includegraphics[width=1.25\textwidth]{../out/diagrams/stage3/cd_dispatcher}}
    \caption{Klassendiagramm Dispatcher}
    \label{fig:cd-dispatcher}
\end{figure}


\section{Datenmodellierung}\label{sec:datenmodellierung}

%% Bestimmen Sie das Datenmodell und dokumentieren Sie es hier mit Hilfe von UML Klassendiagrammen
%% unter Beachtung der Designprinzipien. Die Modelle können mit Hilfe eines UML-Tools erstellt werden.
%% Hier ist dann ein Übersichtsbild einzufügen.
%% Geben Sie eine kurze textuelle Beschreibung des Datenmodells
%% und deren wichtigsten Klassen und Methoden an.


\section{Verhaltensmodellierung}\label{sec:verhaltensmodellierung}

\begin{figure}
    \makebox[\textwidth][c]{\includegraphics[height=0.9\textheight]{../out/diagrams/stage3/stm_top_level}}
    \caption{Der Einstiegspunkt einer \gls{anlage}.
    Verwaltet die Betriebszustände und Sub-State Machines}
    \label{fig:stm_top_level}
\end{figure}

Die Top-Level-STM \ref{fig:stm_top_level} stellt die allgemeine Logik einer Anlage dar.
Diese State-Machine dient zur Veranschaulichung des der Logik und wurde nicht direkt implementiert.
Um tiefe Hierarchien in den STMs zu vermeiden und die parallel laufenden STMs einfacher zu verwirklichen wurde
für die Implementation wurden basierend auf der \ref{fig:stm_top_level} STM modelliert.

Die folgenden STMs modellieren das Verhalten der Anlage.
Sie laufen alle parallel.

\begin{figure}
    \makebox[\textwidth][c]{\includegraphics[height=0.9\textheight]{../out/diagrams/stage3/stm_operation_manager}}
    \caption{Steuerung der Betriebszustände mit dazugehörigem Schalten der \gls{ampel} und Fehlerbehandlung}
    \label{fig:stm_operation_manager}
\end{figure}

\begin{figure}
    \makebox[\textwidth][c]{\includegraphics[width=1.25\textwidth]{../out/diagrams/stage3/stm_recieve_workpiece}}
    \caption{Die Annahme eines \glspl{workpiece}s am Anfang der \gls{anlage}}
    \label{fig:stm_recieve_workpiece}
\end{figure}

\begin{figure}
    \makebox[\textwidth][c]{\includegraphics[width=1.25\textwidth]{../out/diagrams/stage3/stm_height_measurement}}
    \caption{Messung der Höhe eines \glspl{workpiece}s}
    \label{fig:stm_hoehe_messen}
\end{figure}

\begin{figure}
    \makebox[\textwidth][c]{\includegraphics[width=1.25\textwidth]{../out/diagrams/stage3/stm_sort_workpiece}}
    \caption{Sortierung von \glspl{workpiece}n}
    \label{fig:stm_sort_workpiece}
\end{figure}

\begin{figure}
    \makebox[\textwidth][c]{\includegraphics[width=1.25\textwidth]{../out/diagrams/stage3/stm_workpiece_transfer}}
    \caption{Transfer eines \glspl{workpiece}s an die nächste \gls{anlage}}
    \label{fig:stm_workpiece_transfer}
\end{figure}

\begin{figure}
    \centering
    \includegraphics[scale = 0.5]{../out/diagrams/stage3/stm_answer_transfer_request}
    \caption{Antworten eines Transfer-Request der vorherigen \gls{anlage}.
    Ein Transfer-Request wird in stm Workpiece Transfer (Abbildung~\ref{fig:stm_workpiece_transfer_request}) gesendet}
    \label{fig:stm_workpiece_transfer_request}
\end{figure}

\begin{figure}
    \makebox[\textwidth][c]{\includegraphics[width=1.25\textwidth]{../out/diagrams/stage3/stm_small_stms}}
    \caption{Kleine STMs}
    \label{fig:stm_small_stms}
\end{figure}

\begin{figure}
    \makebox[\textwidth][c]{\includegraphics[width=1.25\textwidth]{../out/diagrams/stage3/stm_error_listener}}
    \caption{Umgang mit Fehlern}
    \label{fig:stm_error_listener}
\end{figure}



%% Ihre Software muss zur Bearbeitung der Aufgaben ein Verhalten aufweisen.
%% Überlegen Sie sich dieses Verhalten auf Basis der Anforderungen und modellieren
%% Sie das Verhalten unter Verwendung von Verhaltensdiagrammen aus den Vorlesungen.

