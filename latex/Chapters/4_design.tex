\chapter{Design}\label{ch:design}

%% Anmerkung: Die Implementierung MUSS zu Ihrem Design-Modell konsistent sein.
%% Strukturen, Verhalten und Bezeichner im Code müssen mit dem Modell übereinstimmen.
%% Daher ist ein wohlüberlegtes Design wichtig.


\section{Systemarchitektur}\label{sec:systemarchitektur}

%% Erstellen Sie eine Architektur für Ihre Software.
%% Geben Sie eine kurze Beschreibung Ihrer Architektur mit den dazugehörenden Komponenten
%% und Schnittstellen an.
%% Dokumentieren Sie hier wichtige technische Entscheidungen.
%% Welche Pattern werden gegebenenfalls verwendet? Wie erfolgt die interne Kommunikation?

In Abbildung~\ref{fig:cmp} ist die Systemarchitektur mithilfe eines UML Komponentendiagramms
visualisiert.

\begin{figure}[h]
    \centering
    \makebox[\textwidth][c]{\includegraphics[width=1.2\textwidth]{../out/diagrams/stage1/cmp}}
    \caption{Komponentendiagramm}
    \label{fig:cmp}
\end{figure}

\subsubsection{Allgemein}
Es wird eine ereignisbasierte Architektur implementiert.
Das System basiert auf eine ungepufferte ereignisbasierte Kommunikation,
die über den Dispatcher abläuft.
Am Dispatcher sind sämtliche Komponenten des Systems als Clients registriert.
Diese erzeugen oder beziehen \glspl{event}, welche über das Qnet-Protokoll durch QNX-Pulse-Messages versendet werden.

\subsubsection{Dispatcher}
Die Eventkommunikation läuft über den Dispatcher nach dem Publish/Subscriber Messaging Pattern ab.
Die \glspl{event} werden Type-Based gefiltert.
Der Dispatcher stellt das Interface IDispatcher bereit, an den sich die Komponenten als Clients hängen.
Clients sind einerseits Publisher, die \glspl{event} erzeugen.
Andererseits gibt es Subscriber, die sich für bestimmte \glspl{event} registrieren und diese beziehen.
Der Dispatcher hat zudem das Interface Qnet, welches die Eventkommunikation mit der anderen \gls{anlage} via QNX-Pulse-Messages ermöglicht.

\subsubsection{EventRecorder}
Der EventRecorder hängt als Publisher und Subscriber am Dispatcher.
Er abonniert alle \glspl{event} um mittels der \gls{record-fn} ein \gls{protokoll} zu erstellen.
Ein Ablauf kann durch die \gls{replay-fn} simuliert werden,
indem die \glspl{event} aus dem \gls{protokoll} erzeugt und an den Dispatcher geleitet werden.

\subsubsection{HAL}
In der Hardware Abstraction Layer befinden sich zum einem die Sensoren.
Sensoren sind Publisher, die \glspl{event} erzeugen und den Dispatcher zusenden.
Zum anderen gibt es Aktoren, diese abonnieren \glspl{event}, mit denen sie angesteuert werden.

\subsubsection{Logic}
In der Logic befinden sich die Komponenten, welche die Logik für den Systemablauf bereitstellen.
Dazu gehören folgende Komponenten:

\subsubsubsection{Datamodel}
Das Datamodel verwaltet Informationen der \glspl{workpiece}.
Für jeden Abschnitt der \gls{anlage}, welche durch Lichtschranken getrennt sind, gibt es eine Queue.
In den Queues werden \glspl{workpiece} und deren Informationen verwaltet.
Das bereitgestellte Interface IDatamodel ermöglicht den Zugriff auf die Daten.

\subsubsubsection{Control}
Die Control beinhaltet Komponenten, die mit FSMs den Systemablauf steuern.
Dazu gehören die Komponenten MainControl, Stoplight, ErrorHandler und ConnectionManagement.
Die Control hat Zugriff auf die Daten des Datamodel über das Innterface IDatamodel.
Alle Komponenten der Control beziehen zudem Events über den Dispatcher als Subscriber und erzeugen Events als Publisher.



\section{Datenmodellierung}\label{sec:datenmodellierung}

%% Bestimmen Sie das Datenmodell und dokumentieren Sie es hier mit Hilfe von UML Klassendiagrammen
%% unter Beachtung der Designprinzipien. Die Modelle können mit Hilfe eines UML-Tools erstellt werden.
%% Hier ist dann ein Übersichtsbild einzufügen.
%% Geben Sie eine kurze textuelle Beschreibung des Datenmodells
%% und deren wichtigsten Klassen und Methoden an.


\section{Verhaltensmodellierung}\label{sec:verhaltensmodellierung}

%% Ihre Software muss zur Bearbeitung der Aufgaben ein Verhalten aufweisen.
%% Überlegen Sie sich dieses Verhalten auf Basis der Anforderungen und modellieren
%% Sie das Verhalten unter Verwendung von Verhaltensdiagrammen aus den Vorlesungen.

