\chapter{Design}\label{ch:design}

%% Anmerkung: Die Implementierung MUSS zu Ihrem Design-Modell konsistent sein.
%% Strukturen, Verhalten und Bezeichner im Code müssen mit dem Modell übereinstimmen.
%% Daher ist ein wohlüberlegtes Design wichtig.


\section{Systemarchitektur}\label{sec:systemarchitektur}

%% Erstellen Sie eine Architektur für Ihre Software.
%% Geben Sie eine kurze Beschreibung Ihrer Architektur mit den dazugehörenden Komponenten
%% und Schnittstellen an.
%% Dokumentieren Sie hier wichtige technische Entscheidungen.
%% Welche Pattern werden gegebenenfalls verwendet? Wie erfolgt die interne Kommunikation?

In Abbildung~\ref{fig:cmp} ist die Systemarchitektur mithilfe eines UML Komponentendiagramms
visualisiert.

\begin{figure}[h]
    \centering
    \makebox[\textwidth][c]{\includegraphics[width=1.2\textwidth]{../out/diagrams/stage1/cmp}}
    \caption{Komponentendiagramm}
    \label{fig:cmp}
\end{figure}

\begin{figure}
    \makebox[\textwidth][c]{\includegraphics[width=1.25\textwidth]{../out/diagrams/stage2/cd_hal}}
    \caption{}
    \label{fig:cd-hal}
\end{figure}


\section{Datenmodellierung}\label{sec:datenmodellierung}

%% Bestimmen Sie das Datenmodell und dokumentieren Sie es hier mit Hilfe von UML Klassendiagrammen
%% unter Beachtung der Designprinzipien. Die Modelle können mit Hilfe eines UML-Tools erstellt werden.
%% Hier ist dann ein Übersichtsbild einzufügen.
%% Geben Sie eine kurze textuelle Beschreibung des Datenmodells
%% und deren wichtigsten Klassen und Methoden an.


\section{Verhaltensmodellierung}\label{sec:verhaltensmodellierung}

%% Ihre Software muss zur Bearbeitung der Aufgaben ein Verhalten aufweisen.
%% Überlegen Sie sich dieses Verhalten auf Basis der Anforderungen und modellieren
%% Sie das Verhalten unter Verwendung von Verhaltensdiagrammen aus den Vorlesungen.

